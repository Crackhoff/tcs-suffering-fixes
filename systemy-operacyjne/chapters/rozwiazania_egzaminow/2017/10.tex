Defragmentacja to tylko fancy nazwa na organizowanie dysku.

Jeśli dopisujemy coś do pliku w środku dysku, to nie mamy za bardzo innej opcji niż znalezienie kolejnego pustego fragmentu (który może być dość daleko) i wpisanie tam danych.

W wyniku takiego procesu po pewnym czasie działania mamy wiele porozrzucanych po całym dysku plików. Chyba nie trzeba tłumaczyć dlaczego to nie jest najlepszy scenariusz.

W każdym razie, defragmentacja polega na przepermutowaniu systemu plików w taki sposób, aby pliki ponownie tworzyły spójny obszar. Dzięki temu aby odczytać jakiś plik nie musimy aż tak skakać po dysku, czyli oszczędzamy trochę czasu.