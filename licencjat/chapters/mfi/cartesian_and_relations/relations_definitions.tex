\begin{definition}[Relacja]
Dowolny podzbiór iloczynu kartezjańskiego \(R \subseteq x \times y\) nazywamy \textbf{relacją}.
\end{definition}
Na relacjach możemy zdefiniować intuicyjne operacje. W~poniższych definicjach przyjmujemy, że mamy już dane jakieś relacje \(R \subseteq A \times B\) i~\(S \subseteq B \times C\).
\begin{definition}[Złożenie relacji]
\begin{equation*}
    S \composition R
        \coloneqq \set{\pars{x, z} \in A \times C : \exists_{y \in B} \pars{x, y} \in R \land \pars{y, z} \in S}
\end{equation*}
Uwaga na kolejność!
\end{definition}
\begin{definition}[Relacja odwrotna]
\begin{equation*}
    R^{-1} \coloneqq \set{\pars{y, x} \in B \times A : \pars{x, y} \in R}
\end{equation*}
\end{definition}
\begin{definition}[Projekcje]
\begin{align*}
    \lproj{R} &\coloneqq \set{x \in A : \exists_{y \in B} \pars{x, y} \in R}\\
    \rproj{R} &\coloneqq \set{y \in B : \exists_{x \in A} \pars{x, y} \in R}\\
\end{align*}
\end{definition}