\subsection{Operacje}

Uzbrojeni w schemat rekursji prostej możemy zdefiniować znane nam działania na liczbach naturalnych.

Same zaś liczby oraz indukcję definiujemy w osobnym pytaniu -- \ref{mfi:nat_and_induction}

\subsubsection{Dodawanie}

Definiujemy:
\begin{itemize}
    \item \( A = Z = \natural \)
    \item \( g(n) = n \)
    \item \( h(p, n, a) = p' \)
\end{itemize}

Dostajemy
\[
    \begin{cases}
        f(0, n) = g(n) = n \\
        f(k', n) = h(f(k, n), k, n) = f(k, n)'
    \end{cases}
\]

Dodawanie definiujemy jako \( a + b = f(a, b) \)

Własności \( f \) możemy zapisać nieco intuicyjniej:
\[
    \begin{cases}
        0 + n = n \\
        1 + n = n' \\
        (k + 1) + n = (k + n) + 1
    \end{cases}
\]


\subsubsection{Mnożenie}
Definiujemy:
\begin{itemize}
    \item \( A = Z = \natural \)
    \item \( g(n) = 0 \)
    \item \( h(p, n, a) = p + a \)
\end{itemize}

Dostajemy
\[
    \begin{cases}
        f(0, n) = g(n) = 0 \\
        f(k', n) = h(f(k, n), k, n) = f(k, n) + n
    \end{cases}
\]

Mnożenie definiujemy jako \( a \cdot b = f(a, b) \)

Własności \( f \) możemy zapisać jako:
\[
    \begin{cases}
        0 \cdot n = n \\
        (k + 1) \cdot n = k \cdot n + n
    \end{cases}
\]

\subsubsection{Potęgowanie}
Definiujemy:
\begin{itemize}
    \item \( A = Z = \natural \)
    \item \( g(n) = 1 \)
    \item \( h(p, n, a) = p \cdot a \)
\end{itemize}

Dostajemy
\[
    \begin{cases}
        f(0, n) = g(n) = 1 \\
        f(k', n) = h(f(k, n), k, n) = f(k, n) \cdot n
    \end{cases}
\]

Mnożenie definiujemy jako \( b^a = f(a, b) \)

Własności \( f \) możemy zapisać jako:
\[
    \begin{cases}
        n^0 = 1 \\
        n^{k + 1} = n^k \cdot n
    \end{cases}
\]
