W tym przypadku treść pytania była zbyt długa by ją wpisać jako tytuł rozdziału, więc w pełnej formie przytoczymy ją dopiero tutaj.

\begin{question}[Losowy spacer na płaszczyźnie] Pionek znajduje się w punkcie \((0, 0)\) na płaszczyźnie. W \(i\)-tym kroku \((i > 1)\) losowany jest kąt \(\alpha_i\) jednostajnie na przedziale \([0, 2\pi]\) (i niezależnie od poprzednich losowań), a pionek przesuwa się ze swojej aktualnej pozycji o wektor jednostkowy wyznaczony przez kąt \(\alpha_i\) (z osią OX). Jaki jest oczekiwany kwadrat odległości od punktu \((0, 0)\) po \(n\) krokach?
\end{question}