 \begin{theorem}[Twierdzenie Dilwortha]
      Jeśli długość maksymalnego antyłańcucha w posecie wynosi $k$, poset ten da się pokryć w całości z użyciem $k$ łańcuchów.
    \end{theorem}

    \begin{proof}
        Robimy indukcję po liczbie elementów posetu; gdy poset \(P\) składa się z jednego elementu, twierdzenie zachodzi w trywialny sposób. W dalszej części dowodu, pisząc \(P\) będziemy mieli na myśli zbiór, na którym zdefiniowany jest nasz poset (bo formalne poset to tupla).
        
        Załóżmy teraz, że mamy poset zdefiniowany na \(n\) elementach. Wiemy, że jego antyłańcuch maksymalny ma długość \(k\). Antyłańcuchów maksymalny spełniający ten warunek nie musi być jeden; oznaczymy zbiór wszystkich antyłańcuchów maksymalnych w tym posecie jako \(A\).
        
        Zdefiniujmy teraz (dla danego antyłańcucha maksymalnego \( \alpha \in A\) zbiory \(U_{\alpha}\) i \(D_{\alpha}\), które będziemy określać odpowiednio jako \textit{upset} i \textit{downset} antyłańcucha \(\alpha\). Do zbioru \(U_{\alpha}\) należą wszystkie elementy \(P\), takie że są (ostro) większe od jakiegokolwiek elementu z \(\alpha\). Do zbioru \(D_{\alpha}\) należą zaś wszystkie elementy \(P\), takie że są mniejsze (ostro) od jakiegokolwiek elementu z \(\alpha\). Bardziej formalnie:
        \begin{equation*}
            U_{\alpha} = \{\,x \in P \mid \exists_{y \in \alpha} \;  y \leq x \} \setminus \alpha
        \end{equation*}
        \begin{equation*}
            D_{\alpha} = \{\,x \in P \mid \exists_{y \in \alpha} \; y \geq x \} \setminus \alpha
        \end{equation*}

        Pierwsza obserwacja którą należy wykonać, to taka że dla dowolnego \(\alpha \in A\) jest tak, że  \(U_{\alpha} \cup D_{\alpha} \cup \alpha = P\). Jest to oczywiste; jeśli istniałby jakiś element z \(P\) który nie należałby ani do downsetu \(\alpha\), ani do upsetu \(\alpha\), ani do antyłańcucha maksymalnego \(\alpha\), to z faktu że nie należy ani do downsetu ani do upsetu wynikałoby, że musiałby należeć do antyłańcucha maksymalnego \(\alpha\) (bo nie jest porównywalny z żadnym jego elementem).
        
        Druga obserwacja: dla dowolnego \(\alpha \in A\) nie istnieje element, który należy jednocześnie do \(U_{\alpha}\) i \(D_{\alpha}\). Aby dowieść tę obserwację, załóżmy nie wprost, że istnieją jakieś \(x, y, z \in P\) takie, że:
        \begin{enumerate}
            \item \(y, z \in \alpha\)
            \item \( x \geq y\)
            \item \( x \leq z\) 
        \end{enumerate}
        
        Wówczas otrzymujemy, że \(y \leq x \leq z\), a więc z tranzytywności w posetach dostalibyśmy, że \(y \leq z\). To prowadziłoby do sprzeczności, bo założyliśmy że \(y, z \in \alpha\), a więc znajdują się w jednym antyłańcuchu (i nie mogą być porównywalne).

        Udowodnimy teraz szybki lemacik.
        
        \begin{lemma}
            \label{dilworth-lemma-1}
            Następujące warunki są równoważne:
            
            \begin{enumerate}
                \item Antyłańcuch maksymalny \(\alpha\) jest taki, że \(D_{\alpha} = \emptyset\);
                \item Antyłańcuch maksymalny \(\alpha\) składa się \textbf{jedynie} ze wszystkich elementów minimalnych posetu \(P\). 
            \end{enumerate}
        \end{lemma}
        \begin{proof}
            \begin{enumerate}
                \item \( (1) \implies (2) \); stosujemy dowód nie wprost. Załóżmy, że istnieje taki antyłańcuch maksymalny \(\alpha\), że \(D_{\alpha} = \emptyset\), ale do \(\alpha\) nie należy jakiś element minimalny z \(P\)\footnote{Należy również uważać na to, że nie możemy w tym dowodzie założyć \textit{czegokolwiek} o elementach w \(\alpha\) -- w szczególności \textit{a priori} możliwe jest, że \(\alpha\) zawiera jakieś elementy które nie są minimalne w \(P\).}. Nazwijmy go \(x\). Rozważmy zbiór \( \alpha' = \alpha \cup \set{x}\). Jeśli \(\alpha'\) jest antyłańcuchem, to znaczy że \(\alpha\) nie był antyłańcuchem maksymalnym i otrzymujemy sprzeczność z założeniami. Jeśli \(\alpha'\) nie jest antyłańcuchem, to oznacza że istnieje jakieś \(y \in \alpha\) takie, że \( y \leq x\) lub \(y \geq x\). 
                
                Nie może być tak, że \(y \leq x\), bo \(x\) jest elementem minimalnym w \(P\). Jeśli \(y \geq x\), to z kolei mamy, że \(x \in D_{\alpha}\), skąd otrzymujemy sprzeczność.
                
                Należy tutaj dodać, że ten dowód nie wprost pokazał jedynie, że \(\alpha\) w takim przypadku zawiera wszystkie elementy minimalne z \(P\), ale nie pokazaliśmy że \textit{nie należą do niego} inne elementy z \(P\). Na szczęście, wiedząc że wszystkie elementy minimalne z \(P\) znajdują się w \(\alpha\), wiemy że jakikolwiek inny element nie może się tam znaleźć (bo skoro nie jest minimalny to jest porównywalny z jakimś minimum, a więc nie należy do antyłańcucha). To już kończy dowód.
                
                \item \( (2) \implies (1) \); element minimalny to taki, że nie istnieje element który byłby od niego mniejszy. Z definicji \(D_{\alpha}\) musi zatem być tak, że \(D_{\alpha} = \emptyset\). 
            \end{enumerate}
        \end{proof}
        
        Niemal identycznym dowodem można posłużyć się, by dowieść następujący lemat:
        
        \begin{lemma}
            \label{dilworth-lemma-2}
            Następujące warunki są równoważne:
            
            \begin{enumerate}
                \item Antyłańcuch maksymalny \(\alpha\) jest taki, że \(U_{\alpha} = \emptyset\);
                \item Antyłańcuch maksymalny \(\alpha\) składa się \textbf{jedynie} ze wszystkich elementów maksymalnych posetu \(P\). 
            \end{enumerate}
        \end{lemma}

        Teraz musimy rozpatrzyć trzy przypadki:
        \begin{enumerate}
            \item \( \exists_{\alpha \in A} \; U_{\alpha} = D_{\alpha} = \emptyset \) \\ 
            W tym przypadku istnieje antyłańcuch maksymalny, którego upset i downset są puste. Nietrudno pokazać, że jest to jedyny antyłańcuch maksymalny (ale to nie ma znaczenia dla dowodu). Co ma znaczenie dla dowodu to to, że wystarczy z każdego elementu tego antyłańcucha utworzyć jednoelementowy łańcuch zawierający tylko siebie samego. Jako że \(\alpha\) ma \(k\) elementów, dostajemy podział \(P\) na \(k\) antyłańcuchów.
            
            \item \( \exists_{\alpha \in A} \; U_{\alpha} \not = \emptyset \wedge D_{\alpha} \not = \emptyset\) \\
            Rozpatruję sobie posety na zbiorach \(B = A \cup U_{\alpha}\) i \(C = A \cup D_{\alpha}\). Jako, że \(U_{\alpha} \not = \emptyset\) i \(D_{\alpha} \not = \emptyset\), to z pewnością \( |B| < |P|\) i \( |C| < |P|\). W takim razie, \(B\) i \(C\) z założenia indukcyjnego da się podzielić na \(k\) łańcuchów.
            
            Ponadto, każdy element \(\alpha\) (zarówno w pokryciu łańcuchowym zbioru \(B\), jak i \(C\)) należy do łańcucha innego niż jakikolwiek inny element \(\alpha\), jako że 2 elementy z jednego antyłańcucha nie mogą znaleźć się w jednym łańcuchu. W dodatku, z definicji zbiorów \(U_{\alpha}\) i \(D_{\alpha}\) bezpośrednio wynika, że każdy element \(\alpha\) jest elementem najmniejszym w odpowiednim łańcuchu z pokrycia łańcuchowego zbioru \(B\), i elementem największym w odpowiednim łańcuchu ze zbioru \(C\). W takim razie po prostu ,,sklejam'' łańcuchy z \(B\) i \(C\) w danym elemencie z \(\alpha\) i mam poprawne pokrycie łańcuchowe całego posetu \(P\).
            
            \item Przypadek przeciwny do dwóch wcześniejszych; \(\forall_{\alpha \in A } \; U_{\alpha} = \emptyset \hspace{5pt} \mathbf{ALBO} \hspace{5pt} D_{\alpha} = \emptyset\) \\

            Korzystając z lematów \ref{dilworth-lemma-1} i \ref{dilworth-lemma-2}, wiemy, że dla każdego \(\alpha \in A\) jest tak, że składa się (jedynie) ze wszystkich elementów maksymalnych lub ze wszystkich elementów minimalnych w \(P\). Nawiasem mówiąc, to bezpośrednio implikuje, że w tym przypadku \( |A| \leq 2 \), ale nie jest to specjalnie ważne. 
            
            Weźmy z \(P\) takie \(x, y\), że \(x\) jest elementem maksymalnym, a \(y\) jest elementem minimalnym w \(P\), przy czym chcemy, by te dwa elementy były porównywalne (tzn. \( x \geq y\)). Taka para dwóch elementów szczęśliwie zawsze istnieje -- jeśli w \(A\) istnieje antyłańcuch bez downsetu, to parę tę stanowi dowolne maksimum i jego świadek bycia w upsecie; analogicznie w dualnym przypadku.  Rozważmy więc poset \(P' = P \setminus \set{x,y} \). Zauważmy, że:
            \begin{itemize}
                \item \( |P'| = |P| - 2\), co pozwala nam zastosować założenie indukcyjne;
                \item jako, że każde \( \alpha \in A\) zawierało zbiór wszystkich elementów maksymalnych lub minimalnych \(P\), wiemy że długość \textbf{wszystkich} antyłańcuchów maksymalnych zmniejszyła się o 1, a więc długość najdłuższego antyłańcucha wynosi \(k-1\).
            \end{itemize}
            To oznacza, że z założenia indukcyjnego \( P' \) możemy podzielić na \(k-1\) łańcuchów. Tak więc dodając do \(P'\) łańcuch \( \{x, y\} \) otrzymujemy podział \(P\) na \(k\) łańcuchów, a to kończy dowód.
            
    \end{enumerate}
        
     \end{proof}