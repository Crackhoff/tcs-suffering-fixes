  \begin{theorem}[Zasada włączeń i wyłączeń]
        \begin{equation}
           |\bigcup_{i \in [n]} A_i| = \sum_{\emptyset \not= X \subset [n]} (-1)^{|X| - 1} \cdot | \bigcap_{i \in X} A_i | 
        \end{equation}
  \end{theorem}

    \begin{proof}
        Weźmy sobie jakieś $x \in \bigcup A_i$ należące dokładnie do $k$ zbiorów (gdzie $k \leq n$). Bez straty ogólności można założyć, że $x \in A_1, A_2, A_3, \dots, A_k$ oraz $x \not\in A_{k+1}, A_{k+2}, A_{k+3}, \dots, A_n$.
        Zauważamy, że $x$ w sumie występującej po lewej stronie postulowanej równości zostanie zliczone raz (oczywiste), a po prawej:
        \begin{enumerate}
            \item Sumując $A_1, A_2, A_3, \dots, A_k$ zliczone zostanie k razy,
            \item Sumując $A_1 \cap A_2, \dots$ zliczone zostanie $\binom{k}{2}$ razy i odjęte od wcześniejszej wartości
            \item Sumując $l$-elementowe przecięcia zbiorów $A_1, A_2, \dots, A_k$ zostanie odjęte/dodane $\binom{k}{l}$ razy.
        \end{enumerate}
        Mamy więc, że $x$ jest zliczone $\binom{k}{1} - \binom{k}{2} + \binom{k}{3} - \binom{k}{4} \dots$ razy. Powołujemy się wtedy na magiczny wzór, który wygląda następująco:
        \begin{equation*}
            \sum_{i=0}^{k}(-1)^{i+1}\binom{k}{i} = -(1 - 1)^k = 0 
        \end{equation*}
        Skąd mamy 
        \begin{equation*}
            \sum_{i=1}^{k}(-1)^{i+1}\binom{k}{i} = 1
        \end{equation*}
        bo $\binom{k}{0} = 1$. To prowadzi nas do konkluzji, że $x$ po prawej stronie również został zliczony 1 raz, czyli wszystko działa tak jak powinno.

     \end{proof}