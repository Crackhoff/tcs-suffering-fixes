Przestrzeń wektorowa złożona z wszystkich wielomianów stopnia co najwyżej \(n\), oznaczana jako \(\Pi_n\), ma wiele różnych baz.

,,Najprostszą'' możliwą bazą jest taka złożona z jednomianów \(1, x, x^2, \dots, x^{n}\) (które są efektywnie wektorami jednostkowymi); wykazanie że taki zbiór w istocie jest bazą jest trywialne. 

Z tego też płynie od razu wniosek, że wymiar takiej przestrzeni wektorowej wynosi \(n+1\). 

Jest też inna baza tej przestrzeni wektorowej, nieco mniej trywialna: 

\(1, 1+x, 1+x+x^2, 1+x+x^2+x^3, \dots\)