\begin{fact}
Niech \(v \in \vfield\). Wówczas \( \norm{\lambda v} = \lambda \norm{v}\).

\end{fact}
\begin{proof}
\[
\norm{\lambda v} = \sqrt{\dotpro{\lambda v}{\lambda v}} = \sqrt{\lambda^2 \dotpro{v}{v}} = \lambda \sqrt{\dotpro{v}{v}} = \lambda \norm{v}
\]
\end{proof}
\begin{fact}
Niech \(v \in \vfield\). Wówczas następujące warunki są równoważne:

\begin{enumerate}
    \item \(\norm{v} = 0\)
    \item \(v = 0\)
\end{enumerate}

\end{fact}
\begin{proof}
Przekształcamy pierwszy warunek równoważnie: 
\begin{align*}
    \norm{v} = 0 \\
    \sqrt{\dotpro{v}{v}} = 0 \\ 
    \dotpro{v}{v} = 0 \\
    v = 0
\end{align*}

Ostatnie przejście wynika bezpośrednio z definicji iloczynu skalarnego.
\end{proof}

\begin{fact}
Niech \(v_1, v_2 \in \vfield\). Wówczas:

\begin{enumerate}
    \item \(\norm{v_1 + v_2} \leq \norm{v_1} + \norm{v_2}\) (szczególny przypadek nierówności Minkowskiego)
    \item \( | \dotpro{v_1}{v_2} | \leq \norm{v_1} \norm{v_2} \)
\end{enumerate}

\end{fact}

\begin{fact}
     \( \frac{\dotpro{v_1}{v_2}}{\norm{v_1} \norm{v_2}} \in [-1, 1] \)
\end{fact}
\begin{proof}
Dostajemy za darmo z faktu, że \( | \dotpro{v_1}{v_2} | \leq \norm{v_1} \norm{v_2} \).
\end{proof}

\begin{definition}[Odległość w przestrzeni euklidesowej]
\textbf{Funkcję odległości} \(d: V \times V \rightarrow \real \) w przestrzeni euklidesowej definiujemy następująco: 

\[ 
    d(v_1, v_2) = \norm{v_1 - v_2} 
\]

Fakt, że jest to poprawnie zdefiniowana metryka bezpośrednio wynika z definicji normy i powyżej przytoczonego szczególnego przypadku nierówności Minkowskiego.
\end{definition}

\begin{definition}[Kąt nieskierowany między dwoma wektorami w przestrzeni euklidesowej]
    Dla pary niezerowych wektorów \(v_1, v_2\) definiujemy funkcję \textbf{kąta nieskierowanego} jako funkcję \(\varphi: V \times V \rightarrow [0, \pi]\) taką, że:
    \[
        \varphi(v_1, v_2) = \arccos(\frac{\dotpro{v_1}{v_2}}{\norm{v_1}\norm{v_2}})
    \]

    Poprawna definicja funkcji \(\varphi\) jako kąta wynika między innymi z faktu, że funkcja \(\arccos\) jest iniektywna na przedziale \([0, \pi]\)
\end{definition}

\begin{definition}[Ortogonalność]
Mówimy, że wektory \(v_1, v_2\) są \textbf{ortogonalne}, jeżeli \(\varphi(v_1, v_2) = \frac{\pi}{2}\).
\end{definition}
\begin{fact}
Jeżeli wektory \(v_1, v_2\) są ortogonalne, to \( \dotpro{v_1}{v_2} = 0\).
\end{fact}
\begin{proof}
\[
    \varphi(v_1, v_2) = \frac{\pi}{2} \iff \frac{\dotpro{v_1}{v_2}}{\norm{v_1}\norm{v_2}} = 0 \implies \dotpro{v_1}{v_2} = 0
\]
\end{proof}

\begin{theorem}
Niech \(\vfield = (V, \oplus, \odot, \field)\) będzie przestrzenią wektorową. Wówczas jeśli zbiór \( A \subseteq V\) jest taki, że dla każdej pary wektorów \(v_1, v_2 \in A\) jest tak, że wektory te są ortogonalne, to zbiór \(A\) jest liniowo niezależny.
\end{theorem}
\begin{proof} 
Jako że kąt między wektorami jest zdefiniowany jedynie dla wektorów niezerowych, to oczywiście zakładamy że wektory zerowe nie należą do \(A\).
Dowód nie wprost; załóżmy że istnieje takie \(v \in A\), że: 

\[ 
    v = \sum_{i=1}^{n} \lambda_i v_i
\]
gdzie \(v_i \in A \setminus \set{v}\).

co przeczyłoby liniowej niezależności.

Wiemy jednocześnie, że dla każdego \(k\) jest tak, że \(\dotpro{v}{v_k} = 0\). 

Rozpiszmy sobie ten iloczyn wektorowy dla określonego \(k\):
\begin{align*}
    0 = \dotpro{v}{v_k} &= \dotpro{\sum_{i=1}^{n} \lambda_i v_i}{v_k} \\ 
    &= \sum_{i=1}^{n} \lambda_i \dotpro{v_i}{v_k} \\ 
    &= \lambda_k \dotpro{v_k}{v_k} + \sum_{i \not = k}^{n} \lambda_i \dotpro{v_i}{v_k} \\ 
    &= \lambda_k \dotpro{v_k}{v_k} + \sum_{i \not = k}^{n} \lambda_i \cdot 0 \\ 
    &= \lambda_k \dotpro{v_k}{v_k} + 0 \\ 
\end{align*}

Jako, że do \(A\) nie należą wektory zerowe, to mamy że dla dowolnego \(k\) jest tak, że \(\lambda_k = 0\), skąd:

\[ 
    v = \sum_{i=1}^{n} 0 \cdot v_i = 0
\]

ale \(v\) miało nie być zerowe (bo \(v \in A\)). Sprzeczność.

\end{proof}

\begin{definition}[Baza ortogonalna]
Bazę przestrzeni wektorowej \(\vfield\) nazwiemy \textbf{ortogonalną}, jeśli składa się z wektorów które są parami ortogonalne.
\end{definition}

\begin{definition}[Wektor unormowany]
Mówimy, że wektor \(v \in V\) jest \textbf{unormowany}, jeżeli \(\norm{v} = 1\).
\end{definition}
\begin{fact}
Każdy wektor można przemnożyć przez skalar \(\lambda\) taki, że w rezultacie otrzymamy wektor unormowany. 
\end{fact}
\begin{proof}
Niech \(v \in V\) będzie wektorem. Wówczas zauważmy, że \(\lambda = \frac{1}{\norm{v}}\) spełnia postulowaną własność, bo:
\[
    \norm{\lambda \cdot v} = \lambda \norm{v} = \frac{1}{\norm{v}} \cdot \norm{v} = 1
\]
\end{proof}


