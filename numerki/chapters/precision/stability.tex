\section{Uwarunkowanie zadania i stabilność algorytmu}
Dana jest liczba t, należy wyznaczyć wartość pewnej funkcji \( f(t) \). \\
\textbf{Współczynnik uwarunkowania} to taka liczba \( A = A(f,t) \), dla której
\[
    \norm{f(t) - f(t^*)} \leq  A\cdot\norm{t - t^*},
\]
\( A \approx f'(t) \) dla rozsądnych funkcji \( f \). \\

\noindent
Dane jest równanie \( g(x) = 0 \), należy wyznaczyć \( x \). \\
\textbf{Współczynnik uwarunkowania} to taka liczba \( A = A(g,x) \), dla której
\[
    \norm{x - x^*} \leq  A\cdot\norm{g(x) - g(x^*)},
\]
\( A \approx \frac{1}{g'(x)} \) dla rozsądnych funkcji \( f \). \\
Problem jest dobrze uwarunkowany, jeśli małe zaburzenia
danych wejściowych spowodują małą zmianę wyniku. \\

\noindent
\textbf{Współczynnik uwarunkowania macierzy A} to \( \kappa = \norm{A} \cdot \norm{A^{-1}} \). Dla równania \( Ax^* = b^* \) z zaburzoną wartością \( b \)
\[
    \frac{\norm{x-x^*}}{\norm{x}} \leq \kappa(A) \cdot \frac{\norm{b-b^*}}{\norm{b}}
\]
Układy równań z wysokim \( \kappa \) są trudne do rozwiązania numerycznie. \\

\noindent
Dane jest \( x \) oraz algorytm obliczający \( A(x) \), które przybliża \( f(x) \). Algorytm jest \textbf{numerycznie poprawny}, jeśli dla każdego \( x \) istnieje \( x^* \) które dobrze przybliża \( x \) oraz \( A(x^*) \) dobrze przybliża \( f(x) \).
Algorytm jest \textbf{numerycznie stabilny}, jeśli małe błędy na wejściu
lub podczas działania algorytmu powodują małe zmiany wyniku.
