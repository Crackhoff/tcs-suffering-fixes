\section{Kodowanie liczb w komputerze}
\begin{itemize}
    \item \textbf{Przez ułamki} - tylko liczby wymierne, dodawanie szybko zwiększa liczbę cyfr w liczniku i mianowniku
    \item \textbf{Stałoprzecinkowe} - ograniczony zakres i marne wykorzystanie pamięci
    \item \textbf{Zmiennoprzecinkowe}
    \[
        x = \plusminus (1+M) \cdot 2^w,
    \]
    gdzie \( w \) to wykładnik (cecha), a \( M \in (0, 1) \) to mantysa. Wykładnik zapisuje się z przesunięciem (\( 2^{b_w-1} - 1 \)). \\
    Kodowanie zera: \;\quad\quad\quad\quad\quad\quad\( w = 0\dots0, \;M=0\dots0 \) \\
    Kodowanie nieskończoności: \;\quad\( w = 1\dots1, \;M=0\dots0 \) \\
    Kodowanie NaN: \;\quad\quad\quad\quad\quad\;\;\;\( w = 1\dots1, \;M \in (0, 1) \) \\
    Kodowanie nieznormalizowane: \( w = 0\dots0, \;M \in (0, 1) \) - liczby mniejsze od \( 2^{-b_M} \)
\end{itemize}