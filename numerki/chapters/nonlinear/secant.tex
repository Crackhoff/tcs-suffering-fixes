\section{Metoda siecznych}
Pomocna, kiedy nie wiadomo, jak obliczyć pochodną. \\
\( x_{n} = x_{n-1} - \frac{f(x_{n-1}) \:\cdot\: (x_{n-1} - x_{n-2})}{f (x_{n-1}) - f(x_{n-2})} \) \\
Zbieżność ponadliniowa z \( r = \frac{1 + \sqrt{5}}{2} \)