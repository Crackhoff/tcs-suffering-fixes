\begin{theorem} [Kroneckera-Capellego (Rouché-Capellego)]
Niech \( Ax = b \) jest układem \( m \) równań z \( n \) niewiadomymi. Oznaczamy \( r_{a} = \text{rank}\: A \), \( r_{ab} = \text{rank}\: [A|b] \). Układ ma rozwiązanie wtw \( r_{a} = r_{ab} \) i jest ono jednoznaczne wtw \( r_{a} = n \). Inaczej rozwiązania stanowią przestrzeń \linebreak o rozmiarze \( n - r_{a} \). \\
Jeśli układ ma dwa rozwiązania, to ma ich nieskończenie wiele. Skoro \( Ax = b \) i \( Ax' = b \), \linebreak to \( A \cdot (\alpha x + (1 - \alpha)x') = b \).
\end{theorem} \\