\section{Metoda QR}
Zaczynając od \( A_0 = A \), stosujemy algorytm:
\begin{itemize}
\onehalfspacing
    \item \( (Q_k, R_k) = \text{decompose}(A_k) \)
    \item \( A_{k+1} = R_kQ_k \)
\end{itemize}
Kolejne \( A_k \) zmierzają do macierzy diagonalnej z wartościami własnymi na przekątnej. W teorii jest tak, jeśli \( A \) jest diagonalizowalna, symetryczna i ma różne wartości własne. \\
Żeby przyspieszyć obliczenia, sprowadzamy \( A \) do macierzy Hessenberga (zera pod przekątną), np odbiciami Householdera i tak unikamy wielokrotnego czasochłonnego rozkładu. \\
Żeby przyspieszyć zbieżność, stosujemy przesunięcie:
\begin{itemize}
\onehalfspacing
    \item \( (Q_k, R_k) = \text{decompose}(A_k - I \cdot \mu) \)
    \item \( A_{k+1} = R_kQ_k + I \cdot \mu \)
\end{itemize}
Za \( \mu \) można przyjąć prawy-dolny element \( A \).