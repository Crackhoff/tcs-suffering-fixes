\section{Potęgowanie macierzy}
Macierz \( A \) jest diagonalizowalna (\textit{non-defective}), jeśli ma \( n \) liniowo niezależnych wektorów własnych. Wtedy \( A = P^{-1}DP \), gdzie \( P \) jest odwracalna, \( D \) diagonalna (z wartościami własnymi). Każda macierz (nawet niediagonalizowalna) ma postać Jordana \( A = P^{-1}JP \), wtedy \( J \) jest jak \( D \) \( + \) jedynki pod przekątną. Z tego wynika, że przy potęgowaniu macierz zachowuje się jak największa wartość własna – jeśli \( \lambda > 1 \), ciąg \( A^k \) jest rozbieżny, jeśli \( \lambda < 1 \) jest zbieżny. \\
\textit{Dygresja}: Jeśli \( P \) jest odwracalna, to \( P^{-1}AP \) ma te same wartości własne, co \( A \) (macierze są sprzężone).