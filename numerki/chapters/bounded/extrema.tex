\section{Ekstrema warunkowe}
Niech \( f: \mathbb{R}^n \rightarrow \mathbb{R} \) będzie funkcją celu. Chcemy zminimalizować (zmaksymalizować) \( f(x) \) na pewnym zbiorze domkniętym zadanym przez ograniczenia \( g_i(x) = 0 \), dla \( i = 1, 2, \dots, k \), gdzie \( g_i: \mathbb{R}^n \rightarrow \mathbb{R} \). Czyli celem jest znaleźć ekstrema warunkowe. Definiujemy funkcję:
\[
    f_{\lambda}(x,\lambda) = f(x) - \sum_{i}\; \lambda_i \cdot g_i(x)
\]
Ekstrema warunkowe \( f \) to ekstrema funkcji \( f_{\lambda} \), czyli miejsca, gdzie
gradient \( \nabla f_{\lambda} \) się zeruje.