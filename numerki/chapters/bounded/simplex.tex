\section{Metoda simplex}
Zaczynając od dowolnego wierzchołka obszaru dopuszczalnego, dopóki to możliwe, przesuwamy się do wierzchołka, który ma lepszą wartość funkcji celu. Jeśli rozwiązanie początkowe \( x = 0 \) nie spełnia nierówności, to dodajemy nową zmienną \( w \), którą odejmujemy od prawej strony każdej z nierówności i próbujemy zmaksymalizować (z nadzieją, że się wyzeruje).
\begin{warning}
    Metoda simplex może być wykładnicza i może się zapętlić. Własność stopu zapewnia Reguła Blanda - wybieramy zmienną bazową i niebazową o najmniejszym indeksie.
\end{warning}