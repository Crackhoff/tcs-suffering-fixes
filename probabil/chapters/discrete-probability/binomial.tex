\begin{definition}
    Mówimy, że zmienna losowa \( X \) ma \textbf{rozkład dwumianowy} z parametrami \( n, p \) (Oznaczana poprzez \(B(n, p))\), jeśli
    dla \( j = 0, 1,..., n \): 
    \[
        P(X = j) = \binom{n}{j}p^j(1-p)^{(n-j)}
    \]
\end{definition}

\begin{theorem}
    Niech \( X \) ma rozkład dwumianowy z parametrami \(n, p \). Wtedy
    \[
        \expected{X} = np
    \]
\end{theorem}

\begin{proof}
    \begin{align*}
    \expected{X} 
        &= \sum_{j=0}^nj\binom{n}{j}p^j(1-p)^{(n-j)} \\
        &= \sum_{j=0}^nj\frac{n!}{j!(n-j)!}p^j(1-p)^{(n-j)} \\
        &= np\sum_{j=1}^n\frac{(n-1)!}{(j-1)!(n-j)!}p^{(j-1)}(1-p)^{(n-j)} \\
        &= np\sum_{k=0}^{n-1}\frac{(n-1)!}{k!(n-1-k)!}p^{k}(1-p)^{(n-1-k)} \\
        &= np\sum_{k=0}^{n-1}\binom{n-1}{k}p^{k}(1-p)^{(n-1-k)} \\
        &= np
    \end{align*}
\end{proof}

\begin{theorem}
    Niech \( X \) ma rozkład dwumianowy z parametrami \(n, p \). Wtedy
    \[
        \variance{X} = np(1-p)
    \]
\end{theorem}

\begin{proof}
    \[
    \variance{X} = \expected{X^2} - (\expected{X})^2
    \]
    Pozostaje nam tylko policzyć \( E[X^2]\)
    
    \begin{align*}
        \expected{X^2} 
            &= \sum_{j=0}^n\binom{n}{j}p^j(1-p)^{n-j}j^2 \\
            &= \sum_{j=0}^n\frac{n!}{j!(n-j)!}p^j(1-p)^{n-j}(j^2-j+j) \\
            &= \sum_{j=0}^n\frac{n!(j^2-j)}{j!(n-j)!}p^j(1-p)^{(n-j)} +\sum_{j=0}^n\frac{n!j}{j!(n-j)!}p^j(1-p)^{(n-j)} \\
            &= n(n-1)p^2 \sum_{j=0}^n\frac{(n-2)!}{(j-2)!(n-j)!}p^{j-2}(1-p)^{(n-j)}\\
            &+np\sum_{j=1}^n\frac{(n-1)!}{(j-1)!(n-j)!}p^{(j-1)}(1-p)^{(n-j)} \\
        &= n(n-1)p^2 + np
    \end{align*}
    
    W takim razie
    \[ \variance{X} = n(n-1)p^2 + np - (np)^2 = np - np^2 = np(1-p) \]
\end{proof}

\begin{theorem}
    MGF:
    \[ M_X(t) = (1 - p + pe^t)^n\]
\end{theorem}

\begin{proof}
    \[
    M_X(t) = \expected{e^{tX}} = \sum_{j=0}^n \binom{n}{j}p^j(1-p)^{n-j}e^{tj} =
    \]
    \[
    =  \sum_{j=0}^n \binom{n}{j}(pe^t)^j(1-p)^{n-j} =
    \]
    \[
    = ((1-p) + pe^t)^n
    \]
\end{proof}