\begin{theorem}
    Ustalmy \( \lambda \).
    Niech \( X_n \) będzie rozkładem dwumianowym z parametrami \( n,  p = \frac{\lambda}{n} \)
    
    Wtedy
    \[
        \lim_{n \rightarrow \infty} P(X_n = k) = e^{-\lambda} \cdot \frac{\lambda^k}{k!}
    \]
\end{theorem}
\begin{proof}
    \begin{align*}
        P(X_n = k) 
            &= \binom{n}{k} \cdot p^k \cdot (1-p)^{n-k} \\
            &= \frac{(n-k+1)\cdot \dots \cdot n}{n^k} \cdot \frac{n^k}{k!} \cdot p^k \cdot (1-p)^n \cdot (1-p)^{-k} \\
            &= \frac{(np)^k}{k!} \cdot \pars{1 - p}^{n} \cdot
            \frac{(n-k+1)\cdot \dots \cdot n}{n^k} \cdot (1-p)^{-k} \\
            &= \frac{\lambda^k}{k!} \cdot \pars{1 - \frac{\lambda}{n}}^{n} \cdot \frac{(n-k+1)\cdot \dots \cdot n}{n^k} \cdot \pars{1-\frac{\lambda}{n}}^{-k} \\
    \end{align*}
    Biorąc granicę dostajemy:
    \begin{align*}
        \lim_{n \rightarrow \infty} P(X_n = k) 
            &= 
              \lim_{n \rightarrow \infty} \pars{ \frac{\lambda^k}{k!} \cdot \pars{1 - \frac{\lambda}{n}}^{n} \cdot \frac{(n-k+1)\cdot \dots \cdot n}{n^k} \cdot \pars{\frac{n - \lambda}{n}}^{-k}} \\
            &= \pars{\lim_{n \rightarrow \infty} \frac{\lambda^k}{k!} \cdot \pars{1 - \frac{\lambda}{n}}^{n}}
                \cdot
                \pars{
                \lim_{n \rightarrow \infty}\frac{(n-k+1)\cdot \dots \cdot n}{n^k} \cdot \pars{\frac{n}{n - \lambda}}^{k}
                } \\
            &= e^{-\lambda} \cdot \frac{\lambda^k}{k!}
    \end{align*}
\end{proof}

Pokażemy teraz trudniejszy dowód, który jest w książce

Szanujemy swoich Czytelników i udowodnimy prosty lemat, z którego skorzystamy.
\begin{lemma}
    Dla dowolnego \( \abs{x} \leq 1 \)
    \[
        e^x(1 - x^2) \leq 1 + x \leq e^x
    \]
\end{lemma}
\begin{proof}
    Korzystamy z rozwinięcia Taylora dla \( e^x \)
    \[
        e^x = \sum_{n=0}^\infty \frac{x^n}{n!}
    \]
    \begin{enumerate}
        \item Zacznijmy od górnej nierówności. W oczywisty sposób
        \[
            1 + x \leq 1 + x + \frac{x^2}{2!} + \dots
        \]
        \item Drugą udowadniamy korzystając z faktu, że 
        \[
            e^x = \sum_{n=0}^\infty \frac{x^n}{n!} \leq \sum_{n=0}^\infty x^n = \frac{1}{1-x} = \frac{1 + x}{1 - x^2}
        \]
    \end{enumerate}
\end{proof}
\begin{lemma}
    Dla \( n \geq 1 \) oraz \(x \geq -1\) zachodzi
    \[
        (1 + x)^n \geq 1 + nx
    \]
\end{lemma}

\begin{theorem}[Twierdzenie 5.5 P\&C]
    Ustalmy \( \lambda \).
    Niech \( X_n \) będzie rozkładem dwumianowym z parametrami \( n,  p = \frac{\lambda}{n} \)
    
    Wtedy
    \[
        \lim_{n \rightarrow \infty} P(X_n = k) = e^{-\lambda} \cdot \frac{\lambda^k}{k!}
    \]
\end{theorem}
\begin{proof}
    Korzystamy z obu poprzednich lematów aby dostać kolejne oszacowania górne:
    \begin{align*}
        P(X_n = k) 
            &= \binom{n}{k} \cdot p^k \cdot (1-p)^{n-k} \\
            &\leq \frac{n^k}{k!} \cdot p^k \cdot \frac{(1-p)^n}{(1-p)^k} \\
            &\leq \frac{n^k}{k!} \cdot p^k \cdot \frac{\pars{e^{-p}}^n}{1 - pk} \\
            &=e^{-pn}\cdot\frac{(np)^k}{k!}\cdot\frac{1}{1 - pk} \\
            &=e^{-\lambda} \cdot \frac{\lambda^k}{k!} \cdot \frac{1}{1 - pk}
    \end{align*}
    Wygląda dość znajomo. Widzimy, że w granicy to wyrażenie zbiega do
    \[
        e^{-\lambda} \cdot \frac{\lambda^k}{k!}
    \]
    
    To teraz oszacowanie dolne
    \begin{align*}
        P(X_n = k)
            &= \binom{n}{k} \cdot p^k \cdot (1-p)^{n-k} \\
            &\geq \frac{(n - k + 1)^k}{k!} \cdot p^k \cdot (1-p)^{n - k} \\
            &\geq \frac{(n - k + 1)^k}{k!} \cdot p^k \cdot (1-p)^n \\
            &\geq \frac{((n-k+1)p)^k}{k!} \cdot \pars{e^{-p}\cdot(1 - p^2)}^n \\
            &\geq e^{-pn} \cdot \frac{(np - p(k+1))^k}{k!} \cdot (1 - np^2) \\
            &= e^{-\lambda} \cdot \frac{(\lambda - p(k + 1))^k}{k!} \cdot (1 - p\lambda)^2
    \end{align*}
    Tutaj ponownie widzimy, że skoro \( p \rightarrow 0 \) to to wyrażenie zbiega do
    \[
        e^{-\lambda} \cdot \frac{\lambda^k}{k!}
    \]
    
    Z twierdzenia o trzech ciągach wychodzi że granica rozkładu dwumianowego to Poisson, niesamowite.
\end{proof}