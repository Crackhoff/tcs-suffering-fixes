\begin{definition}
    Niech $\mu, \nu$ będą rozkładami prawdopodobieństwa nad skończonym zbiorem $S$. Normą całkowitego wahania (total variation distance) tych rozkładów nazywamy wartość
    $$ \left\|\mu-\nu\right\|_{TV} = \max_{A \subseteq S} \left|\mu\left( A  \right) - \nu\left( A \right) \right| .$$ 
\end{definition}

\begin{lemma}
    Niech $\mu, \nu$ będą rozkładami prawdopodobieństwa nad skończonym zbiorem $S$. Niech $B = \left\{ x \in S : \mu\left( x  \right) \ge \nu\left( x \right)  \right\} $. Zachodzi
    $$ \left\|\mu-\nu\right\|_{TV} = \mu\left( B  \right) - \nu\left( B  \right) = \nu\left( B^{c} \right) - \mu\left( B^{c} \right)  .$$ 
\end{lemma}
\begin{proof}
    Niech $A \subseteq S$. Zachodzi $\mu\left( A  \right) - \nu\left( A  \right) \le \mu\left( A \cap B  \right) - \nu\left( A \cap B  \right) \le \mu\left( B  \right) - \nu\left( B  \right) $, gdzie pierwsza nierówność to wzięcie tylko tych elementów $A$ dla których różnica rozkładów jest dodatnia a potem rozszerzamy się na wszystkie takie elementy. Analogicznie mamy $\nu\left( A  \right) - \mu\left( A \right) \le \nu\left( B^{c} \right) - \mu\left( B^{c} \right) = 1-\nu\left( B  \right) - 1 + \mu\left( B  \right) $. Zatem te wartości są równe i świadczą o maksimum.
\end{proof}

\begin{lemma}
    Niech $\mu, \nu$ będą rozkładami prawdopodobieństwa nad skończonym zbiorem $S$. Zachodzi
    $$ \left\|\mu-\nu\right\|_{TV} = \frac{1}{2} \sum_{x \in S}^{} \left|\mu\left( x  \right) - \nu\left( x \right) \right| .$$ 
\end{lemma}
\begin{proof}
    Z poprzedniego lematu dostajemy, że dla $B = \left\{ x \in  S : \mu\left( x  \right) \ge \nu\left( x \right)  \right\} $ jest
    $$ \left\|\mu-\nu\right\|_{TV} = \frac{1}{2} \left( \mu\left( B  \right) - \nu\left( B  \right) + \nu\left( B^{c} \right) - \mu\left( B^{c} \right)  \right) = \frac{1}{2} \sum_{x \in S}^{} \left|\mu\left( x  \right) - \nu\left( x  \right) \right|. $$ 
\end{proof}

\begin{lemma}
    Niech $\mu, \nu$ będą rozkładami prawdopodobieństwa nad skończonym zbiorem $S$. Zachodzi
    $$ \left\|\mu-\nu\right\|_{TV} = \frac{1}{2} \sup \left\{ \sum_{x \in  S}^{} f\left( x  \right) \mu\left( x  \right) - \sum_{x \in  S}^{} f\left( x  \right) \nu\left( x  \right) : \max_{x \in S}\left|f\left( x  \right) \right|\le 1 \right\}  .$$ 
\end{lemma}
\begin{proof}
    $\left( \ge  \right) $ Jeśli $\max_{x \in S }\left|f\left( x  \right) \right|\le 1$, to mamy 
    \begin{align*}
        \frac{1}{2} \left|\sum_{x \in  S}^{} f\left( x  \right) \mu\left( x  \right) - \sum_{x \in  S}^{} f\left( x  \right) \nu\left( x  \right) \right| &\le  \frac{1}{2} \sum_{x \in S}^{} \left|f\left( x  \right) \left( \mu\left( x  \right) - \nu\left( x  \right)  \right) \right| \\ 
      &\le  \frac{1}{2} \sum_{x \in S}^{} \left|\mu\left( x  \right) - \nu\left( x  \right) \right| = \left\|\mu-\nu\right\|_{TV}.  
    \end{align*}

    $\left( \le  \right) $ Bierzemy $B = \left\{ x \in S : \mu\left( x  \right) \ge \nu\left( x \right)  \right\} $ i definiujemy funkcję
    $$ f^{\star}\left( x  \right)  = \left\{ \begin{array}{lr} 1, & x \in B \\ -1, & x \in B^{c} \end{array} \right. , $$ 
    która daje
    $$ \frac{1}{2} \left( \sum_{x \in S}^{} f^{\star}\left( x  \right) \mu\left( x  \right) - \sum_{x \in S}^{} f^{\star}\left( x  \right) \nu\left( x  \right)  \right) = \frac{1}{2} \sum_{x \in S}^{} \left|\mu\left( x  \right) - \nu\left( x  \right) \right|= \left\|\mu-\nu\right\|_{TV}. $$ 
\end{proof}

\begin{definition}
    Niech $\mu, \nu$ będą rozkładami prawdopodobieństwa nad skończonym zbiorem $S$. Sprzęganiem $\mu$ i $\nu$ nazywamy dowolną parę zmiennych losowych $\left( X,Y \right) $ taką, że $X$ ma rozkład $\mu$, a $Y$ ma rozkład $\nu$. W szczególności te zmienne nie muszą być niezależne.
\end{definition}

\begin{lemma}
    Niech $\left( X,Y \right) $ będzie sprzęganiem $\mu$ i $\nu$. Zachodzi
    $$ \left\|\mu-\nu\right\|_{TV} \le P\left( X \neq Y \right) . $$ 
    Ponadto istnieje sprzęganie dla którego zachodzi równość.
\end{lemma}
\begin{proof}
    Dla dowolnego $A \subseteq S$ mamy
    $$ \mu\left( A  \right) - \nu\left( A  \right) = P\left( X \in A  \right) - P\left( Y \in  A  \right) \le P\left( X \in A \cap Y \notin A  \right) \le P \left( X \neq Y \right) . $$ 
    Analogicznie $\nu\left( A  \right) - \mu\left( A  \right) \le  P\left( X \neq Y \right) $. To daje żądaną nierówność.

    Teraz skonstruujemy sprzęganie spełniające równość. Niech $B = \left\{ x \in S : \mu\left( x  \right) \ge \nu\left( x  \right)  \right\} $. Niech $p_1 = \mu\left( B  \right) - \nu\left( B  \right) $, $p_2 = \nu\left( B^{c} \right) - \mu\left( B^{c} \right) $. Mamy $p_1 = p_2 = \left\|\mu-\nu\right\|_{TV}$. Niech $p_3 = 1-p_1 = 1-p_2$.

    Rzucamy monetą z prawdopodobieństwem orła $p_3$. Jeśli wypadnie orzeł to ustalamy $X = Y = s $, gdzie $s$ wybieramy z $S$ z rozkładem $\left( \frac{1}{p_3} \min\left( \mu\left( s  \right) ,\nu\left( s  \right)  \right) : s \in S \right) $. Jeśli wypadnie reszka ustalamy $X = x $ i $Y=y$, gdzie $x$ jest wybierany losowo z $S $ z rozkładem $\left( \frac{1}{p_1} \max\left( \mu\left( x  \right) -\nu\left( x  \right) , 0 \right) : x \in S \right) $, a $y$ z rozkładem $\left( \frac{1}{p_2} \max\left( \nu\left( x  \right) -\mu\left( x  \right) , 0 \right) : x \in S \right) $. W przypadku reszki jedna zmienna przyjmuje tylko te wartości, na których $\mu$ jest większe, a druga tylko te, na których $\nu$ jest większe. Mamy więc $P\left( X\neq Y \right) = 1-p_3 = \left\|\mu-\nu\right\|_{TV}$, a $\left( X,Y \right) $ faktycznie jest sprzęganiem $\mu$ i $\nu$ -- zmienne mają odpowiednie rozkłady.
\end{proof}

