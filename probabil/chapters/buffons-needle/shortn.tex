\begin{definition}
    Krótka igła jest to piękne zdarzenie które zachodzi gdy \(l < d\) - długość igły jest mniejsza niż szerokość deski.
\end{definition}

\begin{theorem}
    Prawdopodobieństwo że igła przetnie którąś z krawędzi wynosi \(\frac{2l}{4\pi}\)
\end{theorem}

\begin{proof}
    Niech \(X\) oznacz odległość do środka igły do najbliższej krawędzi deski. Oznaczmy \(\theta\) kąt ostry pomiędzy naszą igłą a jedną z wielu równoległych prostych.
    
    Oczywiście obie z tych zmiennych mają rozkład jednostajny.
    \[
        f_X(x) = 
        \begin{cases}
            \frac{2}{d}: & 0 \leq x \leq \frac{d}{2} \\
            0: & \text{wpw}\\
        \end{cases}
    \]
    \[
        f_\theta(\theta) = 
        \begin{cases}
            \frac{2}{\pi}: & 0 \leq \theta \leq \frac{\pi}{2} \\
            0: & \text{wpw}
        \end{cases}
    \]
    Zmienne te są w oczywisty sposób niezależne więc ich wspólny rozkład prawdopodobieństwa będzie ich iloczynem.
    \[
        f_{X\theta}(x, \theta) = 
        \begin{cases}
            \frac{4}{d\pi}: & 0 \leq x \leq \frac{d}{2}, 0 \leq \theta \leq \frac{\pi}{2}  \\
            0: & \text{wpw}
        \end{cases}
    \]
    Igła przecina krawędź gdy \(x \leq \frac{l}{2}\sin{\theta}\).
    
    Liczymy zatem nasze piękne prawdopodobieństwo:
    \[
        \int_{\theta=0}^{\frac{\pi}{2}} \int_{x=0}^{\frac{l}{2}\sin{\theta}} \frac{4}{d\pi} \, dx\, d\theta = \frac{2l}{d\pi}
    \]
    Przeliczenia zostawiamy dla Czytelnika aby sobie powtórzył całkowanie. \#AM
\end{proof}