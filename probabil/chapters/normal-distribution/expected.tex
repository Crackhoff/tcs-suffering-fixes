Najpierw wyliczmy wartość oczekiwaną standardowego rozkładu normalnego.
\begin{theorem}
    Wartość oczekiwana standardowego rozkładu normalnego wynosi 0, wariancja wynosi 1.
\end{theorem}

\begin{proof}
Wartość oczekiwana wynosi 0, ponieważ standardowy rozkład normalny jest symetryczny wobec prostej \( OY\)
Wariancja:
\[
    \variance{Z} = \expected{Z^2} - \expected{Z}^2 = \expected{Z^2} = \\
\]
ponieważ \( \expected{Z} = 0 \)
\[  
    = \frac{1}{\sqrt{2\pi}}\int_{-\infty}^{z}t^2e^{-t^2/2} dt = 
\]
\[
    = \frac{1}{\sqrt{2\pi}}\int_{-\infty}^{z}(t)(te^{-t^2/2}) dt =
\]
całkowanie przez części
\[
    -\frac{1}{\sqrt{2\pi}}te^{-t^2/2}|_{-\infty}^{\infty} + \frac{1}{\sqrt{2\pi}}\int_{-\infty}^{\infty}e^{-t^2/2} dt = 1
\]
Ponieważ pierwszy wyraz jest równy 0 a drugi jest to dystrybuanta na od \(-\infty \) do \( \infty \) więc wynosi ona 1.
\end{proof}

\begin{lemma}
    Zmienna losowa ma rozkład normalny wtedy i tylko wtedy gdy jest transformacją liniową zmiennej losowej o standardowym rozkładzie normalnym
\end{lemma}

\begin{proof}
    Ponieważ zmienna losowa \( X\) z \(N(\mu, \sigma^2) \) ma ten sam rozkład co \( \sigma Z + \mu \) mamy że
    \[\expected{X} = \expected{\sigma Z + \mu} = \mu
    \]
    \[
        \variance{X} = \variance{\sigma Z + \mu} = \sigma^2
    \]
\end{proof}
czyli ten \sout{dzban} dzwon ma efektywnie przesunięcie o \( \mu \).