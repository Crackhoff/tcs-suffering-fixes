\newpage
\section{epsilon-NFA}
\begin{definition}
    Definiujemy \( \epsilon \)-NFA jako NFA, z tym że:
    \[
        \delta: Q \times (\Sigma \cup \set{\eps}) \times Q
    \]
\end{definition}
Bardziej intuicyjnie -- pozwalamy na przechodzenie krawędzią bez konsumowania obecnej litery.
Niestety sprawia to, że przy tej definicji \( \epsilon \)-NFA nie jest jako tako NFA, który zawsze musi użyć obecnej litery.

\begin{definition}
    Definiujemy \(\mathbf{\epsilon}\)\textbf{-domknięcie} zbioru stanów \( B \) jako zbiór \( B^{A, \epsilon} \) taki, że
    \[
        B^{A, \epsilon} = \bigcup_{i=0}^\infty B_i^{A, \epsilon}
    \]
    gdzie
    \[
        B_i^{A, \epsilon} = \begin{cases}
            B & \text{ gdy } i = 0 \\
            \tilde \delta\pars{B_{i-1}^{A, \epsilon}, \epsilon} & \text{ gdy } i > 0
        \end{cases}
    \]
\end{definition}
\begin{definition}
    Niech \(\powerset^{A, \epsilon}(Q) \) to zbiór wszystkich \(\epsilon\)-domkniętych zbiorów stanów.
    Definiujemy \( \Delta :  \)
\end{definition}

\begin{theorem}
    Niech \( L \subseteq \Sigma^* \). Następujące warunki są równoważne:
    \begin{enumerate}
        \item Istnieje DFA rozpoznający L
        \item Istnieje NFA rozpoznający L
        \item Istnieje \(\epsilon\)-NFA rozpoznający L
        \item Istnieje wyrażenie regularne opisujące L
    \end{enumerate}
\end{theorem}
\begin{proof}

    Wcześniej pokazaliśmy już, że (1) \( \iff \) (2).
    Widać, że (2) \(\implies \) (3).
    Możemy jeszcze łatwo pokazać, że (3) \( \implies \) (1) -- konstrukcja DFA wygląda podobnie jak dla NFA, z tym że stanami DFA są \( \powerset^{A, \epsilon}(Q) \), a funkcja przejścia to \( \Delta \)
    
    Pokażmy teraz, że (4) \( \implies \) (3), zrobimy to za pomocą indukcji strukturalnej po zadanym wyrażniu \( \alpha \)
    
    \begin{itemize}
        \item \( \alpha = 0 \)
        \[
            A_0 = \pars{\set{s, f}, \Sigma, \varnothing, \set{s}, \set{f}} 
        \]
        
        \item \( \alpha = a \in \Sigma \)
        \[
            A_a = \pars{\set{s, f}, \Sigma, \set{
            \pars{
            s, a, f
            }
            }, \set{s}, \set{f}} 
        \]
        
        \item \( \alpha = \alpha_1 \cdot \alpha_2 \)
        \[
            A_{\cdot} = \pars{Q_1 \cup Q_2, }
        \]
        
        \item \( \alpha = \alpha_1 + \alpha_2 \)
        
        \item \( \alpha = \alpha_1^* \)
        
         \item \( \alpha = 1 \)
    \end{itemize}
    
    
    Teraz pokażemy, że (1) \( \implies \) (4) co domknie nam równoważności.

    Czynimy obserwację, że jeśli \( w \in L(A) \) to istnieje ścieżka \( s \rightarrow q_f \in F \) idąca stanami \( s = q_1, q_1, \dots, q_{k+1} \in F \)
    
    Definiujemy \( \alpha_{i, j}^k \) jako takie wyrażenie regularne, które reprezentuje wszystkie ścieżki prowadzące od \( q_i \) do \( q_j \), w której stany pośrednie należą do \( \set{q_1, \dots, q_k} \)
    
    \[
        \alpha_{i, j}^0 = \set{a \in \Sigma : (q_i, a, q_j) \in \delta } \cup \beta_{i, j}
    \]
    gdzie
    \[
        \beta_{i, j} = \begin{cases}
            \varnothing & \text{ gdy } i \neq j \\
            \set{\eps}
        \end{cases}
    \]
\end{proof}