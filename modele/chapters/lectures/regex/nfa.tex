\section{Niedeterministyczne Automaty Skończone}
Deterministyczne automaty są dość proste -- jak widzą jakąś literę to po prostu za nią idą.

Ale co gdyby nasz automat miał kilka opcji do wyboru i mógł ,,zgadywać'', albo rozgałęziać się? Wtedy dostajemy automat, który przestaje być deterministyczny, ale nadal można go sensownie zdefiniować.

\begin{definition}
    \textbf{Nieeterministyczny Automat Skończony} (NFA od nondeterministic finite automaton) to tupla:
    \[
        A = (Q, \Sigma, \delta, S, F)
    \]
    gdzie
    \begin{itemize}
        \item \( Q \) jest skończonym zbiorem stanów
        \item \( \Sigma \) jest skończonym alfabetem
        \item \( \delta \subseteq Q \times \Sigma \times Q \) jest relacją przejścia
        \item \( S \subseteq Q \) jest zbiorem stanów startowch
        \item \( F \subseteq Q \) jest zbiorem stanów akceptujących (końcowych)
    \end{itemize}
\end{definition}

To co się zmieniło względem DFA to:
\begin{itemize}
    \item \( \delta \) jest dowolną relacją (czyli niekoniecznie funkcją), co oddaje fakt, że możemy przechodzić do różnych stanów na podstawie tej samej litery
    \item \( S \) jest zbiorem stanów startowych (czyli może być ich więcej niż jeden)
\end{itemize}

Zdefiniujmy pomocniczą funkcję \( \tilde \delta \), która mówi nam dokąd możemy się rozgałęzić jeśli jesteśmy w stanach \( \beta \) i widzimy literę \( a \). Jest to w pewnym sensie stworzenie takiej funkcji jaką jest \( \delta \) w DFA.

\begin{definition}
    \( \tilde \delta : \powerset\pars{Q} \times \Sigma \rightarrow \powerset\pars{Q} \)
    
    \[
        \tilde \delta(\beta, a) 
            = \set{q \in Q \mid \exists_{q_b \in \beta} (q_b, a, q) \in \delta}
    \]
\end{definition}

Podobnie jak w DFA, \( \hat \delta \) definiuje gdzie możemy skończyć jeśli zaczynamy w jakimś zbiorze stanów \( \beta \) oraz mamy słowo \( w \).
\begin{definition}
    \( \hat \delta : \powerset\pars{Q} \times \Sigma^* \rightarrow \powerset\pars{Q} \)
    \[
        \hat \delta(\beta, w) = \begin{cases}
        \beta & \text{ jeśli } w = \eps \\
        \tilde \delta(\hat \delta(\beta, x), a) & \text{ jeśli } w = xa, a \in \Sigma \\
        \end{cases}
    \]
\end{definition}

\begin{definition}
    \[ 
        L(A) = \set{w \in \Sigma^* \mid \hat \delta(S, w) \cap F \neq \varnothing}
    \]
\end{definition}

Można się zastanawiać, czy NFA jest silniejszy niż DFA (tzn. czy może akceptować więcej języków).
Okazuje się, że nie. Co więcej, jesteśmy w stanie przekształcić dowolny NFA w DFA.
\begin{theorem}
    Niech \( L \subseteq \Sigma^* \).
    Następujące warunki są równoważne:
    \begin{enumerate}
            \item Istnieje DFA \( A_D \), którego \( L(A_D) = L \)
        \item Istnieje NFA \( A_N \), którego \( L(A_N) = L \) 
    \end{enumerate}
\end{theorem}
\begin{proof}
    \begin{description}
        \item ,,\( \implies \)''
        
            Każdy DFA jest NFA, bo każda funkcja jest relacją.
            
        \item ,,\( \impliedby \)''
        
        Niech 
        \[ A_N = (Q, \Sigma, \delta, S, F) \]
        
        Korzystamy z faktu, że \( \tilde \delta \) jest elegancką funkcją i konstruujemy:
        \[ A_D = (\powerset(Q), \Sigma, \tilde \delta, S, \mathcal{F}) \]
        gdzie \( \mathcal{F} = \set{\beta \in \powerset(Q) \mid \beta \cap F \neq \varnothing} \)
        
        Dla pełności potrzebujemy pokazać, że \( \hat \delta_{A_N}(S, u) \cap F \neq \varnothing \iff \hat \delta_{A_D}(S, u) \in \mathcal{F} \),
        ale to mamy tak naprawdę z definicji, bo \( \hat \delta_{A_D} = \hat \delta_{A_N} \)
    \end{description}
\end{proof}

Wszystko fajnie, ale nie ma nic za darmo -- NFA są mniejsze (mają mniej stanów).
\begin{lemma}
    Dla każdego \( n \in \natural \) istnieje NFA \( A_N \) o \(n + 1\) stanach, taki, że każdy DFA \( A_D \) dla którego \( L(A_N) = L(A_D) \) ma co najmniej \( 2^n \) stanów.
\end{lemma}
\begin{proof}
    
\end{proof}