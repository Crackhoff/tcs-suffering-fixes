\section{Enumeratory}

\begin{definition}
Enumerator dla zbioru \(S \subseteq \natural \) to taka maszyna Turinga, która ma dedykowaną taśmę ,,wyjściową'' i która wypisuje na nią elementy \(S\) (i tylko elementy \(S\), w dowolnej kolejności), zapisane unarnie i separowane znacznikiem. 

Jeśli \( x \in S \) to \( x \) zostanie wypisane po jakimś skończonym czasie; jeśli \(x \not \in S\), to nigdy nie zostanie wypisane. 

Enumerator nie może cofać głowicy na taśmie wyjściowej, w szczególności modyfikować tego co już napisał.

W celach ujednolicenia zapisu umawiamy się, że liczby zapisywane są unarnie za pomocą symbolu \(0\), a separowane symbolem \(1\). 

Przykładowo, jeśli na fragmencie taśmy wyjściowej enumeratora znajduje się napis
\[
    \dots 1000100100100001 \dots
\]
to znaczy że wypisał on na tym fragmencie liczby \(3\), \(2\), \(2\) oraz \(4\).
\end{definition}

\section{Związki enumeratorów z RE i R}

Każde słowo z dowolnego języka \( L \subseteq \Sigma^* \) możemy traktować jako liczbę w systemie o podstawie \( \card{\Sigma} \) -- będziemy zatem utożsamiać \( L \) z jakimś \( S_L \subseteq \natural \).

Okazuje się że jest prawdziwe twierdzenie:

\begin{theorem}
    Język \( L \) jest rekurencyjnie przeliczalny wtedy i tylko wtedy gdy istnieje enumerator dla \( S_L \)
\end{theorem}
\begin{proof}
    \begin{description}
        \item \( \implies \)
        
        Bierzemy MT \( M \) rozpoznającą \( L \).
        Będziemy symulować kolejne słowa równolegle tj. konstruujemy enumerator \( M' \) który:
        \begin{enumerate}
            \item Tworzy pustą kolejkę symulacji słów
            \item Dla kolejnych \( w \in \Sigma^* \):
            \begin{enumerate}
                \item Dodaje symulację \( w \) do kolejki
                \item Wykonuje po jednym kroku każdej aktywnej symulacji
                \item Jeśli któreś słowo zostało zaakceptowane to wypisz je unarnie na taśmie wyjściowej.
            \end{enumerate}
        \end{enumerate}
        
        \item \( \impliedby \)
        
        Aby sprawdzić czy \( w \in L \) zapisujemy \( w \) unarnie i do znudzenia gapimy się na wyjście enumeratora aż zobaczymy na nim \( w \) -- jeśli \( w \in L \) to enumerator kiedyś wypisze \( w \) i je zaakceptujemy, a jak nie no to nie.
        
    \end{description}
\end{proof}

Mamy też podobne twierdzenie dla języków rekurencyjnych:
\begin{theorem}
      Język \( L \) jest rekurencyjny wtedy i tylko wtedy gdy istnieje enumerator dla \( S_L \) wypisujący elementy \( S_L \) w ustalonym porządku liniowym (np. leksykograficznym). 
\end{theorem}
\begin{proof}
    \begin{description}
        \item \( \implies \)
        
        Bierzemy MT \( M \) z własnością stopu rozpoznającą \( L \).
        Będziemy symulować kolejne słowa w porządku leksykograficznym tj. konstruujemy enumerator \( M' \) który:
        \begin{enumerate}
            \item Dla kolejnych \( w \in \Sigma^* \) symuluje \( M \) na \( w \).
            W końcu się zatrzymamy i jeśli zaakceptowaliśmy to wypisujemy \( w \) na wyjście.
        \end{enumerate}
        
        \item \( \impliedby \)
        
        Aby sprawdzić czy \( w \in L \) zapisujemy \( w \) unarnie i czytamy wyjście enumeratora aż napotkamy \( w \) lub coś większego w zadanym porządku -- wiemy, że \( w \) już później nie wystąpi więc możemy się spokojnie zatrzymać odrzucając \( w \).
        
    \end{description}
\end{proof}