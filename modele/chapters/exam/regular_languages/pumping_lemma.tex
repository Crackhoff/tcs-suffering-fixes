\section{Lemat o pompowaniu} 

\subsection{Twierdzenie}

\begin{lemma}[O pompowaniu]
    Jeśli \( L \) jest językiem regularnym to:
    \[
        \exists_{n > 0} : \forall_{w : \abs{w} \geq n} \exists_{xyz = w} : \abs{xy} \leq n \land \abs{y} \geq 1 \land \forall_{i \in \natural} : xy^iz \in L
    \]
\end{lemma}
\begin{proof}
    Skoro \( L \) jest językiem regularnym to istnieje DFA A, który rozpoznaje L.
    
    Niech \( n = \abs{Q} \) i weźmy dowolne słowo \( w \) dla którego \( m = \abs{w} \geq n \).
    
    Skoro słowo jest akceptowane to istnieje ścieżka \( s = q_0, \dots q_m \in F \)
    
    Mamy więc \( m + 1 > n \) stanów, czyli jakiś stan musi się powtarzać. 
    Z takich stanów wybieramy \( q_i \), którego pierwsze powtórzenie jest najwcześniej. Niech \( q_j \) będzie drugim wystąpieniem stanu \( q_i \)
    
    Mamy zatem \( x = a_0\dots a_{i-1} \), \( y = a_i\dots a_{j-1} \), \( z = a_j \dots a_{m-1} \)
    
    Oczywiście \( \abs{xy} \leq n \) bo inaczej \( q_i \) nie powtarzałby się najwcześniej.
    
    Czynimy teraz fajną obserwację, mianowicie skoro  \( q_i = q_j \) to słowo \( y \) może wystąpić dowolną (również zerową) liczbę razy w akceptowanym słowie.
    
    Innymi słowy, dla dowolnego \( i \in \natural \) \( xy^iz \in L \)
\end{proof}

Proszę pamiętać, że jest to warunek konieczny, ale \textbf{nie jest on wystarczający} by język był regularny. Istnieją języki które zdecydowanie nie są regularne, a spełniają warunki przedstawione w lemacie o pompowaniu. 

\subsection{Przykładowe zastosowania}

Pokażemy najbardziej klasyczny przykład znany ludzkości. 

\begin{theorem}
    Język
    \[
        L = \set{a^nb^n : n \in \natural}
    \]
    nie jest językiem regularnym.
\end{theorem}

\begin{proof}
    Będziemy usiłowali pokazać, że \(L\) \textit{nie} spełnia lematu o pompowaniu. To oznacza, że będziemy musieli pokazać, że zachodzi dla niego zaprzeczenie własności opisanej w tym lemacie. Innymi słowy, chcemy pokazać że:
    
    \[
        \forall_{n>0} \exists_{w \in L} \forall_{xyz=w : |xy| \leq n \land |y| \geq 1} \exists_{i \in N} xy^iz \not \in L 
    \]
    
    Prościej o tym myśleć jak o pewnej grze: nasz przeciwnik daje nam stałą \(n\); my odpowiadmy mu słowem; ten nam je dzieli na 3 części które spełniają warunki lematu, a my możemy ,,wypompować'' część \(y\). Wygrywamy jeśli słowo nie należy do \(L\). Jeśli mamy tutaj strategię wygrywającą, to znaczy że język nie jest regularny. 
    
    W przypadku naszego języka: nasz oponent daje nam jakieś \(n\). Pora się odegrać, odegrać słowem postaci \(a^{2n}b^{2n}\). Słowo to niewątpliwie należy do \(L\), więc mogliśmy to zrobić. Przeciwnik musi je jakoś podzielić, tak by \( w = xyz\), \(|xy| \leq n \) i \(y \geq 1\). Widzimy tutaj bardzo śmieszną rzecz: niezależnie od podziału, \(xy = a^k\) dla jakiegoś \(k\). Wobec tego również \(y = a^l\) dla niezerowego \(l\), które jest mniejsze niż \(n\). W takim razie ,,depompujemy'' \(y\), ustawiając \(i\) na 0 i otrzymując słowo postaci \(a^{2n-l}b^{2n}\), które z pewnością nie należy do języka. Hehe. 
\end{proof}