\section{Operacje na językach bezkontekstowych}

\begin{theorem}
    Języki bezkontekstowe są zamknięte na sumę.
\end{theorem}
\begin{proof}
    Niech \( L_1, L_2 \in CFL \). Mamy zatem gramatyki bezkontekstowe \( G_1, G_2 \)
    
    Dla języka \( L = L_1 \cup L_2 \) konstruujemy gramatykę \( G = G_1 \cup G_2 \) (zakładamy, że zbiory nieterminali tych gramatyk są rozłączne).
    Dodajemy do gramatyki produkcję:
    \[
        S \rightarrow S_1 \mid S_2
    \]
\end{proof}

\begin{theorem}
    Języki bezkontekstowe są zamknięte na konkatenację.
\end{theorem}
\begin{proof}
    Czynimy tak jak w przypadku sumy ale naszą produkcją którą dodajemy jest
    \[
        S \rightarrow S_1S_2
    \]
\end{proof}

\begin{theorem}
    Języki bezkontekstowe są zamknięte na gwiazdkę Kleenego.
\end{theorem}
\begin{proof}
    Jak wyżej, ale naszą produkcją jest 
    \[
        S \rightarrow S_1S \mid \varepsilon
    \]
\end{proof}


\begin{theorem}
    Języki bezkontekstowe \textbf{nie} są zamknięte na przecięcie.
\end{theorem}
\begin{proof}
    Kontrprzykładem jest \( L_1 = \set{a^nb^nc^k : n, k \in \natural} \) \( L_2 = \set{a^kb^nc^n : n, k \in \natural} \).
    Oczywiście \( L_1 \cap L_2 = \set{a^nb^nc^n : n \in \natural} \) co wiemy że nie jest bezkontekstowe.
\end{proof}


\begin{theorem}
    Języki bezkontekstowe \textbf{nie} są zamknięte na dopełnienie.
\end{theorem}
\begin{proof}
    Mamy z praw de Morgana
    \[
        (L_1 \cap L_2) = \complement{(\complement{L_1} \cup \complement{L_2})}
    \]
    Jeśli zatem mielibyśmy zamknięcie na dopełnienie to byśmy mieli zamknięcie na przecięcie co już pokazaliśmy że nie jest prawdziwe.
\end{proof}