\section{Gramatyki kontesktowe}

\begin{definition}
\textbf{Gramatyka kontekstowa} (Context-Free Grammar) to czwórka \( G = (N, \Sigma, P, S) \) gdzie
\begin{itemize}
    \item \( N \) to skończony zbiór zmiennych (nieterminale)
    \item \( \Sigma \) - alfabet (terminale)
    \item \( P \) - produkcje \( P \subseteq  
    (N \cup \Sigma)^* \, N \, 
    (N \cup \Sigma)^* \times 
    (N \cup \Sigma)^* \)
    \item \( S \in N \) - symbol startowy
\end{itemize}
\end{definition}
gdzie każda produkcja jest postaci:
\[
    \alpha_1 A \alpha_2 \rightarrow_G \alpha_1 \gamma \alpha_2
\]
gdzie \( \alpha_1, \alpha_2, \gamma \in (N \cup \Sigma)^* \) oraz \( A \in N \)
oraz \( |\gamma| \geq 1\) \\
Praktycznie wszystko co poznaliśmy z bezkontekstowych znajduje się też tutaj.

\begin{definition}
    \textbf{Język  generowany} przez gramatykę G to oczywiście 
    \[
        L(G) = \set{w \in \Sigma^* \mid S \rightarrow_G^* w}
    \]
\end{definition}

\begin{definition}
    Język jest kontekstowy (Context-Sensitive Language) jeśli jest generowany przez jakąś gramatykę kontekstową.
\end{definition}

\begin{definition}
    \textbf{Gramatyka monotoniczna} to gramatyka \( G = (N, \Sigma, P, S) \), w której
    każda produkcja jest postaci
    \[
        \alpha \rightarrow \beta \\
    \]
\end{definition}
    gdzie \( \alpha, \beta \in (N \cup \Sigma)^* \) oraz \( |\alpha| \leq |\beta|\)

Klasycznie dopuszczamy produkcję
\[
        S \rightarrow \eps
\]
gdy \( S\) nie znajduje się po prawej stronie w żadnej produkcji

Każda gramatyka kontekstowa jest monotoniczna z definicji.\\
Ciekawszy jest:\\
\begin{lemma}
    Dla gramatyki monotonicznej istnieje równoważna gramatyka kontekstowa.

\end{lemma}
\begin{proof}

\end{proof}