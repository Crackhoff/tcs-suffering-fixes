\section{Twierdzenie Immermana-Szelepcsenyi'ego}

W tej sekcji zajmiemy się twierdzeniem w pewnym sensie dualnym do tw. Savitcha pokazanego poprzednio.

\begin{definition}
    Definiujemy problem \textsc{UNREACHABILITY} jako:
    \begin{itemize}
        \item wejście: Nieskierowany graf \( G = (V, E) \), wierzchołki \(v_1\) i \(v_2\).
        \item pytanie: Czy \textbf{nie} istnieje ścieżka z \(v_1\) do \(v_2\)? 
    \end{itemize}
\end{definition}

\begin{theorem}[Immerman-Szelepcsenyi]
    \( \textsc{UNREACHABILITY} \in \textsc{NLOGSPACE} \)
\end{theorem}
\begin{proof}
    Mamy do pokonania trzy przeszkody:
    \begin{itemize}
        \item Mamy logarytmicznie wiele pamięci -- możemy przechowywać jedynie stałą liczbę liczników.
        
        \item Działamy na niedeterministycznej maszynie, musimy więc zapewnić że jeśli jakaś ścieżka obliczeń odpowie ,,TAK'' to istotnie \( v_1 \) i \( v_2 \) nie są osiągalne.
        
        \item Mamy odpowiedzieć ,,TAK'' jeśli ścieżki między \( v_1, v_2 \) \textit{nie ma} czyli wszystkie możliwe spacery długości \( n \) nie osiągają \( v_2 \).
    \end{itemize}
    
    Chcielibyśmy utworzyć sobie niedeterministyczny pomocniczy algorytm
    \[
        \textrm{reachable}\pars{s, t, d, c}
    \]
    który sprawdza, czy istnieje ścieżka długości \(d\) z~\(s\) do \(t\). Wartość \(c\) jest pomocniczą informacją, którą funkcja otrzymuje od wyroczni --- wkrótce dowiemy się, jak ją tam zapakować. Na razie załóżmy, że magicznie znamy \(c\) i~jest to \emph{liczba} wierzchołków osiągalnych w~\(d\) krokach z~\(s\).
    
    De facto będzie to niedeterministyczna maszyna Turinga obliczająca funkcję boolowską. Ale co to tak naprawdę znaczy? Każda gałąź obliczeń musi zwrócić (czyli np. wypisać na taśmę wyjściową) jedną z~trzech wartości:
    \begin{itemize}
        \item ,,TAK''
        \item ,,NIE''
        \item ,,NIE WIEM''
    \end{itemize}
    Któraś gałąź musi zwrócić coś innego niż ,,NIE WIEM'', a~te, które zwracają coś innego, muszą to zwrócić spójnie (czyli wszystkie ,,TAK'' albo wszystkie ,,NIE''). Za odpowiedź maszyny przyjmujemy odpowiedź tych, które wiedzą.
    
    Intuicyjnie, głównym problemem jest to, że nie możemy (tak jak dla \(\textsc{REACHABILITY}\)) zgadnąć spaceru i powiedzieć ,,TAK, ten spacer nie działa'' bo być może źle zgadliśmy. Chcielibyśmy się upewnić, że obliczenia, które odpowiadają ,,NIE'', faktycznie mają ku temu podstawy.
    
    Przedstawimy teraz algorytm i~przekonamy się, że spełnia on powyższe kryteria:
\begin{minted}{python}
def reachable(s, t, d, c):
    k = 0
    for v in V:
        if [you do not want to bother with v]: # nondeterministic choice (A)
            continue
        [choose a path of length d from s] # nondeterministic choice (B)
        if [you did not hit v]:
            return "IDK"
        if v == t:
            return "YES"
        else:
            k++
    if k == c:
        return "NO!"
    else:
        return "IDK"
\end{minted}
    Mamy tu dwa momenty, w~których niedeterminizm rozdziela nasz program na niezależne, lecz spójne obliczenia. Skąd mamy spójność?
    
    Jeśli poprawna odpowiedź to ,,TAK'', to istnieje obliczenie, która \emph{nie zignoruje} \(v = t\) w~kroku (A) i~trafi na niego w~kroku (B). Być może istnieje nawet wiele takich obliczeń i~wszystkie odpowiedzą ,,TAK''.
    
    Skąd jednak wiemy, że żadne obliczenie nie odpowie wtedy ,,NIE''? A~kiedy jest jedyna sytuacja, w~której obliczenie odpowiada ,,NIE''? Wtedy, gdy \(k = c\). A~kiedy tak jest? Tylko wtedy, gdy było \(c\) wierzchołków, których obliczenie nie zignorowało w~kroku (A) i~do których dotarło w~kroku (B), ale \emph{żaden z~nich} nie był \(t\). Ale zaraz --- przecież \(c\) to~liczba wierzchołków osiągalnych z~\(s\) w~\(d\) krokach --- wiemy to od wyroczni. Skoro zatem odwiedziliśmy \emph{wszystkie} takie i~\emph{żaden z~nich} nie był \(t\), to po prostu \(t\) nie jest osiągalne z~\(s\) w~\(d\) krokach, czyli odpowiedź ,,NIE'' jest słuszna i~nikt inny nie mógł odpowiedzieć ,,TAK''.
    
    Ta obserwacja jest kluczowa dla działania algorytmu.
    
    Nie wiem jak dalej :|.
    
\end{proof}