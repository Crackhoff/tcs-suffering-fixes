\section{Proste redukcje dla znanych problemów}



\subsection{Definicje} 

\subsubsection{NAE-Sat (Not-All-Equal SAT)}

\begin{enumerate}
    \item \textsc{IN:} Formuła logiczna w 3-CNF
    \item \textsc{PYTANIE:}  Czy istnieje wartościowanie takie, że dla wszystkich klauzul prawdą jest, że w obrębie klauzuli znajduje się literał wartościowany na \textsc{TAK} jak i literał wartościowany na \textsc{NIE}? 
\end{enumerate}

\subsubsection{EX2SAT}

\begin{enumerate}
    \item IN: Formuła logiczna \(\varphi\) w postaci
    \[  (x_1 \oplus y_1 ) \land (x_2 \oplus y_2) \land ... \land (x_n \oplus y_n)
    \]
    oraz \( K \) liczba naturalna zapisana binarnie.
    \item PYTANIE: : Czy istnieje wartościowanie zmiennych dla \(\varphi\), które spełnia dokładnie
    \(K\) klauzul w \(\varphi\)?
\end{enumerate}

\subsubsection{HDE (Homogenous Diophantine Equation)}

\begin{definition}
Jednorodne równanie Diofantyczne to równanie diofantyczne zdefiniowane przez wielomian, którego wszystkie niezerowe wyrazy mają ten sam stopień. Stopień wyrazu w równaniu Diofantycznym to suma wykładników zmiennych, które w nim występują -- na przykład \( x^5y^6z^1 \) ma stopień 12.  
\end{definition}

\begin{definition}
    Definiujemy problem zero-one HDE następująco: 
    
    \begin{itemize}
        \item \textsc{IN:} Jednorodne równanie diofantyczne stopnia 2, postaci \(P(x_1, x_2, x_3, \dots, x_n) = B\), gdzie \(B \in \natural\). 
    \end{itemize}
    \item \textsc{PYTANIE:} Czy istnieje takie rozwiązanie tego równania, że wartości \( x_1, x_2, \dots, x_n \in \set{0,1}\)?  
\end{definition}

\subsubsection{Bounded-Tiling}

\begin{definition}
    
\end{definition}

\subsubsection{H-Coloring}
Niech \(H = \pars{V_H, E_H}\) będzie grafem bez pętli. Problem \(H\textsc{-COLORING}\) jest następujący:
\begin{itemize}
    \item WEJŚCIE: Graf symetryczny \(G = \pars{V_H, E_H}\) bez pętli
    \item PYTANIE: czy istnieje homomorfizm \(h\) z~\(G\) w~\(H\), czyli
        \begin{gather*}
            h\colon V_G \mapsto V_H\\
            \forall_u\forall_v\pars{\pasr{u, v} \in E_G \implies \pars{h\pars{u}, h\pars{v}} \in E_H}
        \end{gather*}
\end{itemize}
\(H\textsc{-COLORING}\) jest \(\np\)-zupełny np. dla \(H\) będącego trójkątem --- jest to wtedy \(3\)-kolorowanie. Natomiast np. dla \(H\) będącego dowolną ścieżką ten problem jest w \(\p\).
\begin{proof}
    Trywialny.
\end{proof}

\subsubsection{Factoring}
    \begin{itemize}
        \item IN: Liczby naturalne N i M (Zapisane binarnie)
        \item OUT: Istnienie takiego K, że K dzieli N oraz 2 \(\le\) K \(\le\) M
        \item COMPLEXITY: NP\(\cap\)coNP
\end{itemize}


\subsubsection{Inference}
    \begin{itemize}
        \item IN: Dwie formuły Boolowskie \(\varphi_1, \varphi_2\) nad tym samym zbiorem zmiennych
        \item PYTANIE: czy każde wartościowanie spełniające \(\varphi_1\) jest również
wartościowaniem spełniającym dla \(\varphi_2\)
    \end{itemize}



\subsubsection{Min-Inference}

\subsubsection{First-Sat}
\begin{itemize}
    \item IN: formuła \(\varphi\) nad Zmiennymi \(X = \{x_1, ..., x_n\} \)
    \item PYTANIE: czy \(\varphi\) ma minimalne względem porządku leksykograficznego spełniające wartościowanie \(\alpha : X \rightarrow \{0, 1\}\) takie że \( \alpha(x_i) = 1?\) 
\end{itemize}

\subsubsection{Abduction}
\begin{definition}
    Niech \(Y\) to będzie zbiór zmiennych boolowskich. Mówimy, że zbiór literałów \(U\) wyczerpuje zmienne \(Y\), jeśli dla każdej zmiennej z~\(Y\) zawiera tę zmienną \emph{albo} jej zaprzeczenie.
\end{definition}
Problem \(\textsc{ABDUCTION}\) jest następujący:
\begin{itemize}
    \item WEJŚCIE: Spełnialna formuła boolowska \(\varphi\) nad zmiennymi \(V, V' \subsetneq V\) oraz \(x \in V \setminus V'\).
    \item PYTANIE: Czy istnieje zbiór literałów \(U\) wyczerpujący zmienne \(V'\) taki, że
        \[
            \pars{\varphi \land \bigwedge_{u \in U}u} \models x
        \]
\end{itemize}
Pokaż, że problem \(\textsc{ABDUCTION}\) jest w~\(\np^\np\).

\subsubsection{Geography}
Gra \(\textsc{GEOGRAPHY}\) rozgrywana jest przez dwóch graczy na grafie skierowanym \(G = (V ; E)\). W pierwszym ruchu gracz I koloruje pewien wierzchołek \(v\) w \(G\),
gracz II koloruje pewien wierzchołek \(v'\)
taki, że \((v, v') \in E\). W kolejnych ruchach gracze
kolorują kolejne niepokolorowane wierzchołki wyznaczając w \(G\) ścieżkę. Przegrywa gracz,
który nie może wybrać żadnego niepokolorowanego jeszcze wierzchołka.
\\
\\
Problem \(\textsc{GEOGRAPHY}\), to pytanie czy gracz I ma strategię wygrywającą tzn. wygrywa zawsze, niezależnie od wyborów II gracza


\subsubsection{Regexp-Equiv}
    \begin{itemize}
        \item IN: Dwa wyrażenia regularne, \(\alpha\) i \(\beta\)
        \item OUT: Czy L(\(\alpha\)) = L(\(\beta\))?
        \item COMPLEXITY: PSPACE-complete
\end{itemize}

\subsubsection{Quantified k-SAT}





\subsubsection{}

\subsubsection{}