\subsection{Współczynnik dwumianowy}
\begin{theorem}[Wzór rekurencyjny na dwumian Newtona]
	\begin{equation}
		\binom{n}{k} = \binom{n-1}{k-1} + \binom{n-1}{k}
	\end{equation}
\end{theorem}

\begin{proof}
	Moc zbioru zawierającego wszystkie podzbiory k-elementowe zbioru n-elementowego wynosi właśnie \(\binom{n}{k}\). Bierzemy sobie ten zbiór i nazywamy go \(F\). Wybieramy jakikolwiek element ze zbioru \(n\)-elementowego którego zbiór podzbiorów rozważamy i nazywamy go \(x\). Tworzymy dwa zbiory, \(A\) i \(B\): \(A\) zawiera wszystkie zbiory z \(F\) których elementem jest \(x\), \(B\) zawiera wszystkie zbiory z \(F\) do których \(x\) nie należy. Z oczywistych względów zachodzi:
	\begin{equation*}
		A \cup B = F
	\end{equation*}
	\begin{equation*}
		A \cap B = \emptyset
	\end{equation*}
	Zatem jeśli zliczymy ile jest elementów w \(A\), a ile w \(B\), to suma tych liczb jest mocą zbioru \(F\). Zbiorów w \(A\) jest \(\binom{n-1}{k-1}\), bo dla każdego zbioru wiemy już, że jest tam \(x\), a pozostałe elementy możemy wziąć jak chcemy. Zbiorów w \(B\) jest \(\binom{n-1}{k}\), bo po prostu wybieramy \(k\) elementów z elementów które są rózne od \(x\) (a których jest właśnie \(n-1\)).
\end{proof}

\subsection{Liczba Stirlinga II rodzaju}
\begin{theorem}[Wzór rekurencyjny na liczbę Stirlinga II rodzaju]
	\begin{equation}
		\stirling{n}{k} = \stirling{n-1}{k-1} +  k \cdot \stirling{n-1}{k}
	\end{equation}
\end{theorem}

\begin{proof}
	Rozpatrujemy zbiór wszystkich podziałów zbioru \(n\)-elementowego na \(k\) niepustych podzbiorów (mający moc równą \(\stirling{n}{k}\)) i nazwijmy go \(F\). Bierzemy sobie jakiś \(x\) z naszego zbioru \(n\)-elementowego i wykonujemy podział zbioru \(F\) na zbiory \(A\) i \(B\). Zbiór \(A\) zawiera wszystkie elementy \(F\) takie, że zawierają \(\{x\}\), a więc, innymi słowy, jeden z ,,bloków'' danego podziału musi być taki, że tylko \(x\) do niego należy. Zbiór \(B\) to zbiór pozostałych podziałów, a więc wszystkie podziały z \(F\) w których \(x\) występuje w swoim ,,bloku'' razem z jakimiś innymi elementami. Z oczywistych względów zachodzi:
	\begin{equation*}
		A \cup B = F
	\end{equation*}
	\begin{equation*}
		A \cap B = \emptyset
	\end{equation*}
	Łatwo zauważyć, że \(A = \stirling{n-1}{k-1}\), bo mamy zawsze jeden ,,segment'' w którym znajduje się sam \(x\), więc moc zbioru \(A\) jest równa liczbie możliwych podziałów całej reszty elementów, czyli właśnie \(n-1\) elementów na \(k-1\) ,,bloków''. Moc zbioru \(B\) wynosi zaś \(k \cdot \stirling{n-1}{k}\), bo dzielimy sobie wszystkie elementy poza \(x\) na \(k\) ,,bloków'', a potem ,,dorzucamy'' \(x\) do któregoś z powstałych już ,,bloków'' (oczywiście ,,bloki'' przy podziale generowanym przez liczbę Stirlinga II rodzaju są niepuste). Ponieważ bloków do których możemy ,,dorzucić'' \(x\) jest k, otrzymujemy \(k \cdot \stirling{n-1}{k}\). To prowadzi nas już do postulowanej równości.
\end{proof}

\subsection{Liczba Stirlinga I rodzaju}

\begin{theorem}[Wzór rekurencyjny na liczbę Stirlinga I rodzaju]
	\begin{equation}
		\stirlingOne{n}{k} =  \stirlingOne{n-1}{k-1} + (n-1) \cdot \stirlingOne{n-1}{k}
	\end{equation}
\end{theorem}

\begin{proof}
	Stosujemy motyw podobny dla powyższych dowodów. \(F\) to zbiór zawierający wszystkie permutacje na \(n\) elementach które ,,rozbijają się'' na \(k\) cykli. Bierzemy sobie jakiś element \(x\) i rozpatrujemy dwa zbiory, \(A\), \(B\) takie że do \(A\) należą wszystkie permutacje gdzie \(x\) przechodzi na siebie samego, a do \(B\) należą wszystkie permutacje gdzie \(x\) nie przechodzi na siebie samego. Moc zbioru \(A\) to \(\stirlingOne{n-1}{k-1}\), bo \(x\) przechodzący na siebie samego stanowi jeden cykl (więc ,,pozostałość'' permutacji trzeba rozbić na \(k-1\) cykli). Obliczenie mocy zbioru \(B\) jest nieco śmieszniejsze, ale okazuje się że jest równe \((n-1) \cdot \stirlingOne{n-1}{k}\). Wynika to z faktu, że bierzemy sobie wszystkie elementy poza \(x\) i robimy na nich permutacje, które da się podzielić na \(k\) cykli; następnie ten element \(x\) ,,dopychamy'' w jakieś miejsce w jakimś cyklu. Przez ,,dopchnięcie'' mam na myśli sytuację, gdy jakiś element \(y\) przechodził na element \(z\), ale \(x\) ,,dopychamy'' w miejsce \(z\); wtedy \(y\) przechodzi na \(x\), a \(x\) na \(z\). Operacja ,,dopchnięcia'' nie psuje liczby cykli, a \(x\) możemy ,,dopchnąć'' zawsze na \(n-1\) miejsc, co daje nam postulowaną równość.
\end{proof}
