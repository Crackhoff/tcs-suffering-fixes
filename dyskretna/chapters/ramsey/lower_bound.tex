\begin{theorem}[Ograniczenie dolne na symetryczną liczbę Ramseya R(k,k)]
	\begin{equation}
		(\sqrt{2})^{k} < R(k,k)
	\end{equation}
\end{theorem}

\begin{proof}
	Oznaczmy jako \(B\) zbiór wszystkich kolorowań, w których istnieje monochromatyczna (czerwona lub niebieska) klika rozmiaru \(k\). Jako, że zbiór wszystkich kolorowań krawędzi grafu na \(N\) wierzchołkach ma moc \(2^{\binom{N}{2}}\), to jeśli pokażemy że \(|B| < 2^{\binom{N}{2}}\), pokażemy że istnieje takie kolorowanie krawędzi grafu, że nie istnieje monochromatyczna klika rozmiaru \(k\), a więc \(|N| < R(k,k)\).

	Zauważmy, że \( B = \bigcup_{x \in \binom{[N]}{k}} B_x \), gdzie jako \(B_x\) rozumiemy zbiór wszystkich takich kolorowań, że zbiór \(x\) (stanowiącym \(k\) wierzchołków z \([N]\)) stanowi monochromatyczną klikę (tzn. dla dowolnej pary krawędzi z \(x\) krawędzie te mają ten sam kolor).

	Możemy to teraz bardzo brutalnie przeszacować:
	\begin{equation*}
		|B| \leq \sum_{x \in \binom{[N]}{k}} |B_x|
	\end{equation*}

	Będziemy tutaj zliczać wielokrotnie mnóstwo rzeczy, ale to nam wystarczy. To zaś przeszacowujemy jeszcze brutalniej:
	\begin{equation*}
		\sum_{x \in \binom{[N]}{k}} |B_x| \leq \binom{N}{k} \cdot 2 \cdot 2^{\binom{N}{2} - \binom{k}{2}}
	\end{equation*}

	Co tutaj zrobiliśmy? Cóż, mówimy że bierzemy \(k\) punktów z \(N\) punktów, ustawiamy wszystkim krawędziom między nimi jeden z dwóch kolorów, po czym mówimy że wszystkie pozostałe krawędzie mogą mieć absolutnie jakikolwiek kolor. Niewątpliwie będziemy podwójnie zliczać wiele kolorowań, ale niespecjalnie nas to tutaj obchodzi.

	Chcielibyśmy teraz pokazać, że \begin{equation*}
		\binom{N}{k} \cdot 2 \cdot 2^{\binom{N}{2} - \binom{k}{2}} < 2^{\binom{N}{2}}
	\end{equation*}
	równoważnie:
	\begin{equation*}
		\binom{N}{k} < 2^{\binom{k}{2}-1}
	\end{equation*}
	Jednocześnie usiłujemy pokazać ograniczenie dolne, więc \begin{equation*}
		N \leq (\sqrt{2})^k
	\end{equation*}
	Jeśli podniesiemy tę nierówność do \(k\)-tej potęgi
	\begin{equation*}
		N^k \leq 2^{\frac{k^2}{2}}
	\end{equation*}
	Przeszacujmy zatem \(\binom{N}{k}\):
	\begin{equation*}
		\binom{N}{k} = \frac{N!}{k! \cdot (N-k)!} \leq \frac{N^k}{k!}
	\end{equation*}
	Takie przeszacowanie w sumie ma sens, skracamy \(N!\) z \((n-k)!\) i szacujemy potęgę kroczącą przez potęgę. Teraz jeszcze fajnie byłoby zauważyć, że \(k! > 2^{\frac{k}{2}+1}\), skąd mamy:
	\begin{equation*}
		\frac{N!}{k! \cdot (N-k)!} \leq \frac{N^k}{k!} < \frac{N^k}{2^{\frac{k}{2}+1}} \leq \frac{ 2^{\frac{k^2}{2}}}{2^{\frac{k}{2}+1}} = 2^{\frac{k \cdot (k-1)}{2} - 1} = 2^{\binom{k}{2} - 1}
	\end{equation*}
	czyli wyszło to co chcieliśmy żeby wyszło.
\end{proof}
\begin{proof}[Alternatywny dowód probabilistyczny]
	Niech \(N = (\sqrt{2})^{k}\) -- będziemy kolorować zbiór \(\binom{N}{2}\) w sposób jednorodnie losowy.
	Aby to zrobić, kolorujemy każdy element z równym prawdopodobieństwem \(\frac{1}{2}\) na kolor \(1\) lub \(2\) -- wtedy zbiorem zdarzeń
	jest \(2^{\binom{[N]}{2}}\) (zbiór wszystkich kolorowań), a każde z nich możemy otrzymać z równym prawdopodobieństwem.

	Zastanówmy się teraz, jaka jest szansa, że ustalając dowolny podzbiór \(S \subset [N]\) o mocy \(k\), a następnie losując kolorowanie,
	zbiór \(S\) będzie monochromatyczny. Łatwo zauważyć, że prawdopodobieństwo wyniesie \(2 \cdot 2^{-\binom{k}{2}}\) (zawężamy zbiór zdarzeń do kolorowania zbioru \(S\), aby
	otrzymać zbiór zdarzeń o mocy \(2^{\binom{k}{2}}\), a istnieją w nim \(2\) zbiory monochromatyczne).

	Możemy teraz oszacować prawdopodobieństwo (oznaczone \(p\)), że istnieje jakakolwiek zbiór jest monochromatyczny:
	\begin{align*}
		p & \leq 2 \cdot 2^{-\binom{k}{2}} \cdot \binom{N}{k}                                             \\
		  & = 2 \cdot 2^{-\frac{1}{2}k(k-1)} \cdot \frac{{(2^{\frac{k}{2}})}^{\underline{k}}}{k!}         \\
		  & < 2 \cdot 2^{-\frac{k^2}{2}} \cdot 2^{\frac{k}{2}} \cdot 2^{\frac{k^2}{2}} \cdot \frac{1}{k!} \\
		  & = \frac{2^{\frac{k}{2}+1}}{k!} \leq 1
	\end{align*}
	Ale ponieważ \(p < 1\) w zbiorze zdarzeń musi istnieć świadek, że
	\(N\) nie zawiera monochromatycznego zbioru wielkości \(k\), co kończy dowód.
\end{proof}
