\begin{theorem}[Erdősa-Szekeresa]
	Dla dowolnego $k$ istnieje takie $N$, że jeśli $N$ punktów na płaszczyźnie znajduje się w pozycji ogólnej to jakieś $k$ z nich znajduje się w pozycji wypukłej.
\end{theorem}
\begin{proof}
	Po pierwsze musimy zauważyć, że zbiór $k$ punktów jest w pozycji wypukłej wtedy i tylko wtedy gdy dowolny jego podzbiór 4-elementowy również jest w pozycji wypukłej. Wynika to z faktu, że jeśli $k$ punktów nie jest w pozycji wypukłej to mogę striangulować i wskazać 4 takie, że również nie są w pozycji wypukłej, a jeśli $k$ punktów jest w pozycji wypukłej to oczywiste jest, że dowolny ich podzbiór również jest w takiej pozycji.

	Po drugie, dla dowolnych 5 punktów jest tak że wśród nich znajdziemy co najmniej 4 takie, że są w pozycji wypukłej. Dowodzimy to w ten sposób, że sobie je rysujemy i bierzemy otoczkę wypukłą; zakładamy że ma ona 3 punkty, bo inaczej teza już się udowodni, po czym puszczamy linię prostą przez 2 punkty które są wewnątrz trójkąta stanowiącego otoczkę wypukłą. Po jednej stronie tej prostej są jakieś 2 punkty z otoczki, które razem z tymi dwoma punktami przez które przeszła prosta są pozycji ogólnej. Fajnie byłoby chyba to rozrysować żeby dowieść.

	No i teraz już po prostu robimy sobie ogólnego Ramseya $R^{(4)}(k,5;2)$ -- kolorujemy czwórki punktów na pierwszy kolor jeśli są w pozycji wypukłej, na drugi jeśli nie. Ponieważ nie ma 5 takich punktów że każdy ich 4-elementowy podzbiór nie jest w pozycji wypukłej (druga obserwacja) to musi być $k$ takich punktów że ich każdy 4-elementowy podzbiór znajduje się w pozycji wypukłej, co daje nam tezę na mocy pierwszej obserwacji.
\end{proof}