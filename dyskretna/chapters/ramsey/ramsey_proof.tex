\epigraph{W twierdzeniach ramseyowych nie chodzi o jakieś liczby, a o granice Twojej wyobraźni}{\textit{Student TCSu który uwalił egzamin}}
\begin{definition}[Liczby Ramseya]
	Liczbą Ramseya $R^{(p)}(k; \ell_1, \ell_2, \ldots, \ell_k)$ dla $p, k, \ell_i \in \natural_1$, nazywamy najmniejszą taką liczbę $N$,
	że dla każdej funkcji $c: \binom{[N]}{p} \to [k]$ (zwanej ,,kolorowaniem'') istnieje kolor $\alpha \in [k]$ oraz zbiór $S \subseteq [N]$
	spełniające: $\card{S} = \ell_\alpha$ oraz $\binom{S}{p} \subseteq c^{-1}(\alpha)$ --
	dowolny zbiór spełniający drugą z tych własności (dla dowolnego $\alpha$) nazywamy ,,monochromatycznym''.
\end{definition}

\begin{theorem}
	Liczby
	\begin{equation*}
		R^{(p)}(k; l_1,l_2,l_3,\dots,l_k)
	\end{equation*}
	są zdefiniowane poprawnie.
\end{theorem}

\begin{proof}
	Prowadzimy indukcję po $p$, $k$ i $\sum_{i \in [k]} \ell_i$. Sprawdzamy przypadki bazowe:
	\begin{enumerate}
		\item Gdy $p = 1$ poprawność wynika z zasady szufladkowej, $N = (\sum_{i \in [k]} \ell_i - 1) + 1$;
		\item Gdy $k = 1$ trywialnie $R^{(p)}(1; \ell) = \min(p, \ell)$;
		\item Trzeci przypadek bazowy polega na tym, że jeżeli dla jakiegoś $j$, $\ell_j = p$ ($p$ jest to minimalna wartość którą w ogóle może przyjąć jakiekolwiek $\ell_i$, inaczej to by nie miało sensu).
		      Wtedy zachodzi $R^{(p)}(k; \ell_1,\ell_2,\ell_3,\ldots,\ell_j,\ldots,\ell_k) = R^{(p)}(k-1; \ell_1,\ell_2,\ell_3,\ldots,\ell_{j-1}, \ell_{j+1}, \ldots,\ell_k)$.
		      Wynika to z prostej obserwacji -- jeżeli istnieje jakikolwiek element $x \subset \binom{[N]}{p}$ dla którego $c(x) = j$, to
		      z definicji możemy przyjąć $\alpha = j, S = x$ aby otrzymać ,,świadka'' dla danego kolorowania --
		      oznacza to, że jedyne kolorowania, które mogą dowodzić fałszywości tw. Ramseya dla danego $N$ nie
		      przypisują żadnemu podzbiorowi koloru $j$, a co za tym idzie możemy rozważać kolorowanie mniejszą ilością kolorów.
	\end{enumerate}

	Zostaje przypadek, gdzie $p, k \geq 2$ i dla każdego $i$ zachodzi $\ell_i > p$. Wprowadźmy oznaczenie:
	\begin{equation*}
		L_i = R^{(p)}(k; \ell_1,\ell_2, \ldots, \ell_{i-1}, \ell_i - 1, \ell_{i+1}, \dots \ell_k)
	\end{equation*}

	Z założenia indukcyjnego $L_i$ jest zdefiniowane poprawnie (bo zredukowaliśmy sumę $l_i$; formaliści mogą sobie podumać nad indukcją po wielu zmiennych i jak działa).
	Teraz definiujemy sobie pewną \textit{potężną} liczbę służącą jako ograniczenie górne:
	\begin{equation*}
		N = R^{(p-1)}(k; L_1, L_2, L_3, \dots, L_k) + 1
	\end{equation*}
	Ponownie, jest ona poprawnie zdefiniowana z założenia indukcyjnego (bo kolorujemy teraz podzbiory $p-1$-elementowe). Po co ta jedynka na końcu? Zaraz się okaże.
	Niech $c$ będzie dowolnym kolorowaniem zbioru $\binom{[N]}{p}$ na $k$ kolorów -- zdefiniujmy sobie teraz kolorowanie $c'$, które koloruje $\binom{[N-1]}{p-1}$ również na $k$ kolorów.
	$c'$ definiujemy sobie w oparciu o $c$ w niezwykle fascynujący sposób -- $c'(S) = c(S \cup \set{N})$. Pomocne może być tutaj narysowanie tej sytuacji.

	Z definicji $N$ wiemy, że mamy jakieś $j$ oraz jakiś zbiór $S \subset [N-1]$ o mocy $\card{S} = L_j$, który jest monochromatyczny względem kolorowania $c'$.

	Ale z definicji $L_j$ wiemy, że istnieje tu podzbiór $l_1$ lub $l_2$ lub \dots lub $l_j - 1$ lub \dots lub $l_s$ elementów taki, że każdy ich $k$-elementowy podzbiór ma ten sam kolor (w kolorowaniu $c$). Jeśli własność ta zachodzi dla jakiegokolwiek $l_i$ gdzie $i \neq j$, to ta własność nam się przez przypadek właśnie udowodniła (i nawet nie użyliśmy naszego śmiesznego kolorowania).
	Pozostaje nam ciekawszy przypadek, gdy mamy $l_j - 1$ punktów takich, że ich każdy $k$-elementowy podzbiór jest pokolorowany na kolor $j$.

	Ale w tym przypadku z definicji $c'$ otrzymujemy fajną własność -- dowolny zbiór $M \in \binom{S}{p-1}$ ma z definicji ten sam kolor co $S \cup \{N\}$ w kolorowaniu $c$. Czyli skoro w $S$ istnieje podzbiór $M$ o mocy $l_j - 1$ taki, że wszystkie elementy $\binom{M}{p}$ są koloru $j$,
	to po dodaniu $N$ do tego podzbioru otrzymujemy zbiór $l_j$-elementowy spełniający założenia o kolorowaniu. Sparse'owanie tego co się stało może trochę zająć, ale w sumie to udowodniliśmy twierdzenie Ramseya. Fajnie.
\end{proof}

\begin{proof}[Alternatywny szkic dowodu]
	Można ułatwić odrobinę powyższy dowód nie indukując się po liczbie kolorów $k$.
	Najpierw dowodzimy przypadek dla $p = k = 2$ -- dowód opisany jest w sekcji o ograniczeniu górnym liczby $R(s, t)$ przez Erd\h{o}sa-Szekeresa.
	Następnie rozszerzamy dowód dla wszystkich $p$ w sposób analogiczny do tego powyżej. Aby rozszerzyć dowód dla wszystkich $k$ wystarczy zauważyć, że
	$$R^{(p)}(k; \ell_1, \ell_2, \ldots, \ell_k) \leq R^{(p)}(2; \ell_1, R^{(p)}(k-1, \ell_2, \ell_3, \ldots, \ell_k))$$
	Aby to udowodnić, przyjmując prawą stronę nierówności jako $N$ i mając kolorowanie $c: \binom{[N]}{p} \to [k]$, definiujemy kolorowanie na dwóch kolorach
	$c'(x) = \min(c(x), 2)$. Z definicji $N$ musi istnieć poprawne kolorowanie $\ell_1$ elementów na kolor $1$, które dowodzi poprawności $N$,
	lub kolorowanie zbioru o mocy $R^{(p)}(k-1; \ell_2, \ell_3, \ldots, \ell_k)$ składające się wyłącznie z kolorów innych od $1$ --
	wtedy kontynuujemy rozumowanie rekurencyjnie na mniejszej ilości kolorów, co dowodzi poprawności $N$.
	Dowód ten jest na tyle fajny na egzaminie, że pracując tylko na dwóch kolorach tracimy trochę ,,boilerplate'u'' z poprzedniego dowodu,
	a ponadto nie musimy się powtarzać przy dowodzie ograniczenia Erd\h{o}sa-Szekeresa.
\end{proof}

\end{proof}



