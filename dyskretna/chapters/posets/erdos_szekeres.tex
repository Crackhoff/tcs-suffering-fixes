  \begin{theorem}[Lemat Erdősa-Szekeresa o podciągach monotonicznych]
        W ciągu składającym się z $n \cdot m + 1$ liczb naturalnych ($n,m \leq 1$) znajduje się podciąg niemalejący długości co najmniej $n + 1$ lub nierosnący długości co najmniej $m + 1$. 
    \end{theorem}

    \begin{proof}
       Zdefiniujmy sobie porządek częściowy na elementach ciągu. Mówimy, że $a \preceq b$, gdy $b$ występuje później niż $a$ w ciągu oraz $a \geq b$. Łańcuch w takim posecie jest podciągiem nierosnącym naszego ciągu, zaś antyłańcuch musi być podciągiem niemalejącym. Z twierdzenia dualnego do twierdzenia Dilwortha wnioskujemy, że w dowolnym posecie $x \cdot y \geq z$, gdzie $x$ jest długością najdłuższego łańcucha, $y$ najdłuższego antyłańcucha, a $z$ mocą posetu. Prowadzi to nas już zasadniczo do tezy, którą możemy teraz dowieść nie wprost: załóżmy, że istnieje taki ciąg długości $n \cdot m + 1$, w którym każdy podciąg niemalejący ma długość maksymalnie $n$, a nierosnący ma długość maksymalnie $m$. Oznaczałoby to, że najdłuższy łańcuch w naszym wcześniej zdefiniowanym posecie ma długość $m$, a antyłańcuch długość $n$, skąd mielibyśmy że $n \cdot m \geq n \cdot m + 1$, co prowadziłoby do sprzeczności.
     \end{proof}
  