\begin{theorem}[Lemat Erdősa-Szekeresa o podciągach monotonicznych]
	W ciągu składającym się z \(n \cdot m + 1\) liczb naturalnych (\(n,m \leq 1\)) znajduje
	się podciąg niemalejący długości co najmniej \(n + 1\) lub nierosnący długości co najmniej
	\(m + 1\).
\end{theorem}

\begin{proof}
	Zdefiniujmy sobie porządek częściowy na elementach ciągu. Mówimy, że \(a \preceq b\),
	gdy \(b\) występuje później niż \(a\) w ciągu oraz \(a \geq b\). Zauważmy, że łańcuch w
	tak zdefiniowanym posecie jest podciągiem nierosnącym naszego ciągu, zaś antyłańcuch
	musi być podciągiem niemalejącym.
	Z twierdzenia dualnego do twierdzenia Dilwortha wnioskujemy, że w dowolnym posecie
	zachodzi \(\posetwidth(P) \cdot \posetheight(P) \geq \card{P}\).
	Prowadzi to nas już zasadniczo do tezy, którą możemy teraz dowieść nie wprost:
	załóżmy, że istnieje taki ciąg długości \(n \cdot m + 1\), w którym każdy podciąg
	niemalejący ma długość maksymalnie \(n\), a nierosnący ma długość maksymalnie \(m\).
	Oznacza to, że najdłuższy łańcuch w naszym wcześniej zdefiniowanym
	posecie ma długość \(m\), a antyłańcuch długość \(n\), skąd otrzymujemy że
	\(n \cdot m \geq n \cdot m + 1\), co prowadzi nas do sprzeczności.
\end{proof}

