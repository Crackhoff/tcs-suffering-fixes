\begin{theorem}[Twierdzenie Spernera]
	Najdłuższy antyłańcuch \(\mathcal D\) w kracie zbiorów $\mathbb B_n$ ma moc $\binom{n}{\lfloor\frac{n}{2}\rfloor} = \binom{n}{\lceil\frac{n}{2}\rceil}$.
\end{theorem}

\begin{proof}[Dowód przez nierówność LYM]
	Przedstawimy dowody dla $\binom{n}{\lfloor\frac n2\rfloor}$ -- dla sufitu są one analogiczne.
	Jesteśmy w stanie wskazać antyłańcuch takiej długości -- $\binom{[n]}{\lfloor\frac{n}{2}\rfloor}$.
	Wystaczy pokazać więc, że nie istnieje dłuższy antyłańcuch.
	Niech $\mathcal{D}$ będzie antyłańcuchem, wtedy:
	\begin{align*}
		\forall_{0 \leq k \leq n}: \binom{n}{k}                       & \leq \binom{n}{\lfloor\frac{n}{2}\rfloor}                            & \implies \text{(tr. obserwacja)} \\
		1 \geq \sum_{X \in \mathcal{D}} \frac{1}{\binom{n}{\card{X}}} & \geq \frac{\card{\mathcal{D}}}{\binom{n}{\lfloor\frac{n}{2}\rfloor}} & \implies \text{(nier. LYM)}      \\
		\binom{n}{\lfloor\frac{n}{2}\rfloor}                          & \geq \card{\mathcal{D}}                                              & \text{(mnożenie stronami)}
	\end{align*}
\end{proof}

\begin{definition}[Łańcuchy symetryczne]
	Łańcuch $C$ w $\mathbb{B}_n$ nazywamy symetrycznym, jeśli $C=\{X_k, X_{k+1},
		\ldots, X_{n-k}\}$, gdzie $X_k \subset X_{k+1} \subset \ldots \subset
		X_{n-k}$ oraz $|X_i|=i$ dla pewnego $k$. Taki łańcuch narysowany na kracie
	jest symetryczny względem środkowego poziomu.
\end{definition}

\begin{proof}[Dowód tw. Spernera przez łańcuchy symetryczne]
	Rozważmy podział $\mathbb B_n$ na łańcuchy symetryczne. Każdy taki łańcuch
	zawiera dokładnie jeden element ze środkowego poziomu (rozmiaru  $\lfloor \frac
		n2 \rfloor$), a więc podział ma $\binom{n}{\lfloor \frac n2 \rfloor}$
	elementów. Z twierdzenia Dilwortha antyłańcuch nie może mieć więcej niż tyle
	elementów.
\end{proof}
Niestety powyższy dowód nie jest jeszcze kompletny, bo nie wiemy jeszcze czy taki
podział na łańcuchy symetryczne wogóle istnieje. Na szczęście właśnie to pokażemy

\begin{theorem}[Podział na łańcuchy symetryczne, rekurencyjnie]
	Dla każdej kraty boolowskiej $\mathbb B_n$ istnieje jej podział na łańcuchy symetryczne.
\end{theorem}
\begin{proof}
	$\mathbb B_0$ ma jeden element, on sam jest symetrycznym łańcuchem. Mając
	podział $\mathbb B_n$ na symetryczne łańcuchy $\mathcal C$, konstruujemy
	podział $\mathbb B_{n+1}$: dla $C = \{X_k,\ldots, X_{n-k}\}\in\mathcal C$
	łańcuchami w $\mathbb B_{n+1}$ są $C' = \{X_k,X_{k+1},\ldots,
		X_{n-k}\cup\{n+1\}\}$ oraz $C'' = \{X_k\cup\{n+1\},
		x_{k+1}\cup\{n+1\},\ldots, X_{n-k-1}\cup\{n+1\}\}$. Te łańcuchy są symetryczne
	w $\mathbb B_{n+1}$ (pierwszy to zbiory ze środkowych poziomów mające od $k$
	do $n+1-k$ elementów, a drugi od $k+1$ do $n+1-k-1$), a stworzenie takich
	łańcuchów dla wszystkich $C\in\mathcal C$ daje nam podział.
\end{proof}

