\begin{definition}
	Dla zbioru \(\mathcal B \subset \displaystyle\binom{[n]}{k}\) jego cieniem
	dolnym nazywamy zbiór \[\Delta\mathcal B = \set{A : \exists_{B\in\mathcal B,
				x\in B} \ A=B\setminus\set{x}},\] a cieniem górnym nazywamy zbiór
	\[\nabla\mathcal B = \set{A : \exists_{B\in\mathcal B, x\in [n]\setminus B} \
			A=B\cup\set{x}}.\] Elementy cienia odpowiednio tracą lub zyskują jeden element
	-- cień jest obcięciem stożka do najbliższego poziomu.
\end{definition}

\begin{theorem}[O rozmiarze cienia]
	Dla $\mathcal B \subset \displaystyle\binom{[n]}{k}$ zachodzą następujące nierówności:
	\begin{align*}
		\card{\Delta\mathcal B} & \ge \frac k{n-k+1}\card{\mathcal B},   \\
		\card{\nabla\mathcal B} & \ge \frac {n-k}{k+1}\card{\mathcal B}. \\
	\end{align*}
	Z powyższego twierdzenia wynika, że $\card{\Delta\mathcal B} \ge \card{\mathcal B}$ dla $k \ge
		\frac {n+1}2$ oraz $\card{\nabla\mathcal B} \ge \card{\mathcal B}$ dla $k \le \frac {n-1}2$.
\end{theorem}
\begin{proof}
	Zliczamy moc zbioru
	$W = \set{(A,B) : B\in\mathcal B, A\in\Delta\mathcal B, A\subset B}$.
	Jest ona równa $k\card{\mathcal B}$, bo każdy element $\mathcal B$ ma dokładnie $k$
	swoich elementów cienia. Jednocześnie każdy element cienia może mieć co najwyżej
	$n-(k-1)$ swoich nadzbiorów w $\mathcal B$, więc $\card W \le \card{\Delta\mathcal B}(n-k+1)$,
	co dowodzi pierwszej nierówności.
	Analogiczne zliczenie dla górnego cienia (każdy element $\mathcal B$ ma $n-k$
	swoich elementów cienia, element cienia ma co najwyżej $k+1$ elementów
	$\mathcal B$) daje drugą nierówność.
\end{proof}

\begin{theorem}[Twierdzenie Spernera, bis]
	Największy antyłańcuch w $\mathbb B_n$ ma rozmiar $\binom{n}{\lfloor \frac n2
			\rfloor}$.
\end{theorem}
\begin{proof}[Dowód przez cienie]
	Niech $\mathcal A$ będzie antyłańcuchem w $\mathbb B_n$ i niech $\mathcal A_j
		= \mathcal A \cap \binom{[n]}{j}$. Jeśli $i = \min\set{j : \mathcal
			A_j\ne\emptyset}$, to dla $i\le \frac{n-1}2$ zbiór $\mathcal A' = (\mathcal
		A\setminus \mathcal A_i)\cup \ \nabla\mathcal A_i$ ma większą moc od
	$\mathcal{A}$ oraz dalej jest antyłańcuchem -- jeśli coś jest nad cieniem
	górnym $\mathcal A_i$, to jest też nad $\mathcal A_i$, więc $\mathcal A$ nie
	byłby antyłańcuchem. Podobnie, jeśli weźmiemy $k = \max\set{j : \mathcal
			A_j\ne\emptyset}$ i będzie $k\ge \frac{n+1}2$. Możemy więc po kolei przesuwać
	kolejne poziomy bliżej środka kraty. Jeśli $2\nmid n$, to możemy wybrać
	dowolny ze środkowych poziomów, bo nierówności z cieniami na to pozwalają.
\end{proof}

\begin{definition}
	Rodzina zbiorów $\mathcal F$ jest przecinająca się, jeśli $\forall_{X, Y\in \mathcal F} \ X \cap Y \neq \emptyset$.
\end{definition}

\begin{theorem}
	Największa rodzina przecinająca się w $\mathbb B_n$ ma rozmiar $2^{n-1}$.
\end{theorem}
\begin{proof}
	Zauważmy, że dla rodziny przecinającej $\mathcal F$ nie może jednocześnie
	zachodzić $X\in\mathcal F$ i $\overline X\in\mathcal F$. Zatem jest
	$\card{\mathcal F}\le 2^{n-1}$. Przykładem takiej rodziny mogą być wszystkie
	podzbiory $\mathbb B_n$ zawierające $1$.
\end{proof}

\begin{theorem}[Erd\H{o}s-Ko-Rado]
	Niech $\mathcal F\subseteq \binom{[n]}{k}$ będzie przecinająca się i niech
	$2k\le n$. Maksymalny rozmiar $\mathcal F$ to $\binom{n-1}{k-1}$.
\end{theorem}
\begin{proof}
	Najpierw zauważmy, że dla $2k>n$ można wziąć $\mathcal F = \binom{[n]}{k}$, bo
	wszystkie takie zbiory muszą się przecinać.

	Faktyczny dowód zaczniemy, rozważając cykl $\sigma$ elementów
	$[n]$ (tj. permutację o jednym cyklu). Przedziałem $k$-elementowym w $\sigma$
	nazwiemy ciąg $k$ elementów występujących kolejno w $\sigma$, być może zapętlając
	się modulo $n$. Pokażemy, że do $\mathcal F$ może należeć co najwyżej $k$
	przedziałów dla każdego takiego cyklu $\sigma$.
	Załóżmy, że $X=\set{x_1,\ldots,x_k}\in\mathcal
		F$ jest przedziałem w $\sigma.$ Zauważmy, że pary przedziałów, z których
	jeden ma prawy koniec w $x_i$, a drugi ma lewy koniec w $x_{i+1}$ dla $i\in
		[k]$ są jedynymi przedziałami, które mogą należeć do $\mathcal F$ i co
	najwyżej jeden z każdej pary należy do $\mathcal F$ (bo muszą się wzajemnie
	przecinać i przecinać $X$, a warunek $2k\le n$ zapewnia, że
	nie przetną się ,,z drugiej strony''). Zatem zbiór
	$W = \set{(X,\sigma) : X\in\mathcal F, \sigma \text{ cyklem w $[n]$}, X \text{ przedziałem w
				$\sigma$}}$ ma co najwyżej $k(n-1)!$ elementów (po $k$ na każdy cykl, a cyklów jest $(n-1)!$).
	Jednocześnie każdy zbiór z $\mathcal F$ można dopełnić do cyklu, stawiając go
	na początku cyklu i permutując jego elementy i pozostałe elementy, co daje nam
	$\card W = \card{\mathcal F}k!(n-k)!$, zatem $\card{\mathcal F} \le
		\frac{k(n-1)!}{k!(n-k)!} = \binom{n-1}{k-1}$.

	Aby znaleźć rodzinę spełniającą to ograniczenie, można wziąć wszystkie elementy z
	$\displaystyle\binom{[n-1]}{k-1}$ z dorzuconym elementem $n$.
\end{proof}

