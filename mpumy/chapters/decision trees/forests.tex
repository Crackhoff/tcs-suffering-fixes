\section{Lasy losowe}

\subsection{Bootstrapping}
\begin{definition}
Niech \( S = \set{x^{(1)}, \dots, x^{(m)}} \subseteq X \)

\textbf{Próbą bootstrapową} z próbki \( S \) nazywamy dowolny \( m \)-elementowy multizbiór elementów wybranych \( S \) losowanych jednostajnie ze zwracaniem.
\end{definition}

Można wyliczyć, że w takiej próbce znajdzie się ok. 63\% obserwacji z \( S \)

Okazuje się, że jeśli szacujemy jakiś parametr \( \phi \) zgodnie z którym zostały wybrane dane do \( S \) to histogram na próbkach boostrapowych jest podobny do niezależnie utworzonych próbek.

Dzięki bootstrapowaniu możemy wygenerować dużo próbek co jest pomocne gdy nie mamy dostępu do całej populacji.

Ponadto mamy ciekawą własność -- niech \( Z_1, \dots, Z_n \) będą parami niezależnymi zmiennymi z tego samego rozkładu o średniej \( \mu \) i wariancji \( \sigma^2 \).

Mamy wtedy
\[
    \expected{\frac{1}{n}\sum_{i=1}^n Z_i} = \mu
\]
oraz
\[
    \Var\pars{\frac{1}{n}\sum_{i=1}^n Z_i} = \frac{\sigma^2}{n}
\]
dostaliśmy mniejszą wariancję.

Jedyny problem jest taki, że próbki są niezależne pod warunkiem oryginalnej próbki \( S \), ale są zależne względem całej populacji.

\subsection{Bagging - bootstrap aggregate}
Niech \( S_1, \dots, S_B \) będą niezależnymi \( m \)-elementowymi zbiorami treningowymi.
Dostajemy \( B \) hipotez \( \widehat h_1(x), \dots, \widehat h_B(x) \) na podstawie których tworzymy nową hipotezę.

Dla regresji:
\[
    \widehat h(x) = \frac{1}{B} \sum_{i=1}^B \widehat h_i(x)
\]
podobnie dla klasyfikacji
\[
    \widehat h(x) = \argmax_{k \in \set{1, \dots, K}} \frac{1}{B} \sum_{i=1}^B \indicator{\widehat h_i(x) = k}
\]

W praktyce mimo zależności uśrednianie hipotez daje sensowne wyniki.

\subsection{Błąd out-of-bag}

Jak zauważyliśmy, każda próbka zawiera ok. 63\% obserwacji, pozostałe potraktujemy jako dane walidacyjne.

Dla \( x^{(i)} \in S \) definiujemy zbiór próbek do których nie trafił:
\[
    S^{(i)} = \set{b : x^{(i)} \notin S_b}
\]

Tworzymy hipotezę out-of-bag:
\[
    \widehat h_{OOB}\pars{x^{(i)}} = \frac{1}{\card{S^{(i)}}} \sum_{b \in S^{(i)}} \widehat h_b\pars{x^{(i)}}
\]

Błąd dla tej hipotezy nazywamy błędem out-of-bag.

\subsection{Las losowy}
Las losowy powstaje przez utworzenie wielu drzew decyzyjnych, każde na innej próbce bootstrapowej.
Podczas konstrukcji drzewa w każdym węźle losujemy podzbiór cech do którego rozważań się zawężamy.


\subsection{Wady i zalety}

Zalety:
\begin{itemize}
    \item mają mniejszą wariancję niż drzewa decyzyjne -- mniejsze przeuczenie
    \item radzą sobie z brakującymi danymi
    \item dobre dla dużych wielowymiarywych danych
    \item mniej wrażliwe na zmiany w obserwacjach
\end{itemize}

Wady:
\begin{itemize}
    \item mniej intuicyjne
    \item nie zawsze dobre dla regresji
    \item jeszcze dłużej się liczy niż drzewa decyzyjne
    \item ryzyko przeuczenia gdy cechy są silnie skorelowane
\end{itemize}