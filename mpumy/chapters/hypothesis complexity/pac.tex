\section{PAC-nauczalność}

Będziemy zajmowali się klasyfikacją binarną, tj. wejścia \( X \) będą dowolne, ale wyjścia \( Y = \set{0, 1} \)

\begin{definition}
    \textbf{Pojęcie} to dowolna funkcja \( c : X \rightarrow Y \) lub równoważnie dowolny podzbiór \( c \subseteq X \) zawierający pozytywne wejścia.
\end{definition}

\begin{definition}
    \textbf{Klasa pojęć} to dowolny podzbiór \( C \subseteq Y^X \)
\end{definition}

Będziemy się zajmowali różnymi zbiorami hipotez \( H \subseteq Y^X \) -- nie muszą być tym samym co \( C \)

Zakładamy, że próbka \( S = \set{x^{(i)}, \dots, x^{(m)}} \subseteq X \) została wybrana niezależnie z jakiegoś rozkładu \( D \), a etykiety są zadane jakimś (ustalonym, ale nieznanym) pojęciem \( y^{(i)} = c(x^{(i)}) \)

Naszym celem jest znaleźć algorytm, który dla danych wejściowych stworzonych w ten sposób minimalizuje ryzyko
\[
    R(h) = \expected{\indicator{h(x) \neq c(x)}}
\]

\begin{definition}
    Mówimy, że klasa pojęć \( C \) jest \textbf{PAC-nauczalna} jeśli istnieje taki algorytm oraz wielomianowa funkcja \( Q : \real^2 \rightarrow \real \) taka, 
    że dla dowolnego rozkładu \( D \) na danych wejściowych oraz dowolnego pojęcia \( c \)
    a ponadto dowolnych \( \varepsilon, \delta > 0\) jeśli mamy próbkę \( S \) o rozmiarze \( \card{S} \geq Q(\frac{1}{\varepsilon}, \frac{1}{\delta}) \) to nasz algorytm konstruuje hipotezę \( h_S \) dla której:
    \[
        p(R(h_S) \leq \varepsilon) \geq 1 - \delta
    \]
\end{definition}

Innymi słowy -- jeśli wiemy jakie rodzaje pojęć dostajemy to mając wystarczająco dużo danych z dowolnie dużym prawdopodobieństwem mamy dowolnie dużą skuteczność.
