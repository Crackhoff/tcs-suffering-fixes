\section{Funkcje jądrowe}

W praktyce dane rzadko są liniowe, a SVM umie dopasować tylko hiperpłaszczyznę.
Aby sobie z tym poradzić moglibyśmy wprowadzić funkcje bazowe \( \phi \) tak jak miało to miejsce w przypadku regresji liniowej.

Nasz problem jest wtedy postaci
\[
    \min_{w, b} \frac{1}{2} \norm{w}^2 + \frac{C}{m}\sum_{i=1}^m \pars{1 - y^{(i)} \cdot \pars{w^T \phi(x^{(i)}) + b}}
\]
co wiemy że jest równoważne maksymalizacji
\[
    \sum_{i=1}^m \alpha_i - \frac{1}{2} \sum_{i=1}^m \sum_{j=1}^m \alpha_i \alpha_j \cdot y^{(i)}y^{(j)} \cdot \phi(x^{(i)})^T\phi(x^{(j)})
\]
pod warunkiem
\[
    \sum_{i=1}^m \alpha_i y^{(i)} = 0
\]
gdzie \( \alpha_i \in \brackets{0, \frac{C}{m}} \)

Zauważamy, że możemy stworzyć funkcję jądrową \( \kappa \)
\[
    \kappa(x^{(i)}, x^{(j)}) = \phi(x^{(i)})^T\phi(x^{(j)}) = \dotproduct{ \phi(x^{(i)}), \phi(x^{(j)})}
\]

Przykładową funkcją jądrową, która jest fajna jest jądro gaussowskie:
\[
    \kappa(x, z) = \exp\pars{\frac{-\norm{x - z}^2}{2\tau^2}}
\]

\begin{definition}
Dla jądra \( \kappa \) oraz elementów \( x^{(1)}, \dots, x^{(m)} \) definiujemy \textbf{macierz Grama}
\[
    K = \begin{bmatrix}
    \kappa(x^{(1)}, x^{(1)}) & \hdots & \kappa(x^{(1)}, x^{(m)}) \\
    \vdots & \ddots & \vdots \\
    \kappa(x^{(m)}, x^{(1)}) & \hdots & \kappa(x^{(m)}, x^{(m)})
    \end{bmatrix}
\]
\end{definition}

\begin{theorem}[Mercer]
    \( \kappa \) jest funkcją jądrową wtedy i tylko wtedy, gdy dla dowolnych obserwacji macierz Grama jest dodatnio półokreślona.
\end{theorem}
\begin{theorem}
    Następujące warunki są równoważne:
    \begin{itemize}
        \item \( M \) jest dodadnio półokreślona
        \item \( M = R^TR \) dla pewnego \( R \)
        \item wartości własne \( M \) są nieujemne
    \end{itemize}
\end{theorem}

\subsection{Przykłady funkcji jądrowych}
\begin{itemize}
    \item Jądro gaussowskie
    \[
        \kappa(x, z) = \exp\pars{\frac{-\norm{x - z}^2}{2\tau^2}}
    \]
    \item Dla skończonego zbioru \( D \) i jego podzbiorów \( A_1, A_2 \)
    \[
        \kappa(A_1, A_2) = 2^{\card{A_1 \cap A_2}}
    \]
    \item Dla rozkładu prawdopodobieństwa \( p \)
    \[
        \kappa(x, z) = p(x)p(z)
    \]
    \item Dla rodziny rozkładów \( \set{p_i : i \in \natural} \) oraz rozkładu wag \( p(i) \)
    \[
        \kappa(x, z) = \sum_{i \in \natural} p_i(x)p_i(z)p(i)
    \]
    \item Mnożenie przez stałą
    \[
        c\kappa(x, z)
    \]
    \item Mnożenie przez funkcję 
    \[
        f(x)\kappa(x, z)f(z)
    \]
    \item Suma
    \[
        \kappa_1(x, z) + \kappa_2(x, z)
    \]
    \item Przeliczalna suma
    \[
        \sum_{i \in \natural} \kappa_i(x, z)
    \]
    \item Iloczyn
    \[
        \kappa_1(x, z) \kappa_2(x, z)
    \]
    \item Zastosowanie wielomianu o nieujemnych współczynnikach
    \[
        P(\kappa(x, z))
    \]
    \item Zastosowanie funkcji wykładniczej
    \[
        \exp(\kappa(x, z))
    \]
    \item Złożenie z funkcją
    \[
        \kappa(\phi(x), \phi(z))
    \]
    \item Mnożenie przez symetryczną dodatnio półokreśloną macierz
    \[
        \kappa(x, z) = x^TAz
    \]
    \item Suma po współrzędnych
    \[
        \sum_{i=1}^k \kappa_i(x_i, z_i)
    \]
    \item Iloczyn po współrzędnych
    \[
        \prod_{i=1}^k \kappa_i(x_i, z_i)
    \]
\end{itemize}