Na wejściu dostajemy dwie liczby A i B zapisane binarnie, każda o $n$ bitach. Rozbijmy je podobnie jak w algorytmie Karatsuby tylko tym razem z $K = 2^{\frac{n}{3}}$.
$$A = A_2 K^2 + A_1 K + A_0$$
$$B = B_2 K^2 + B_1 K + B_0$$
Potraktujmy teraz A i B jak wielomiany, czyli:
$$A(X) = A_2 X^2 + A_1 X + A_0$$
$$B(X) = B_2 X^2 + B_1 X + B_0$$
Obliczmy teraz wartości wielomianów $A(X)$ i $B(x)$ w punktach $0,1,-1,2,-2$, czyli:
$$A(0) = A_0$$
$$A(1) = A_2 + A_1 + A_0$$
$$A(-1) = A_2 + -A_1 + A_0$$
$$A(2) = 4A_2 + 2A_1 + A_0$$
$$A(-2) = 4A_2 + -2A_1 + A_0$$
Jak widać złożoność obliczania wartości w każdym z tych punktów wynosi $\bigO(n)$, gdyż dodawanie i mnożenie przez 2 i 4 możemy wykonywać w czasie $\bigO(n)$. Wykonajmy również takie same obliczenie dla wielomianu $B(X)$. Skorzystajmy z twierdzenia, że skoro $f,g$ są wielomianami to:
$$f(x) \cdot g(x) = (f\cdot g)(x)$$
A więc jeżeli przez oznaczymy sobie $C(X) = A(X)\cdot B(X)$, ($C(X)$ przy okazji jest wynikiem który chcemy uzyskać) to mamy, że $C(0) = A(0)\cdot B(0), C(1) = A(1)\cdot B(1)$ itd. A więc zauważmy, że każde naszych $A(0),B(0),A(1),...$ ma długość rzedu $\frac{n}{3}$ i obliczmy każde $C(0),C(1),C(-1),...$ rekurencyjnie. A więc jak narazie wykonaliśmy $\bigO(n)$ operacji oraz 5 wywołań rekurencyjnych na liczbach długości $\frac{n}{3}$. Zauważmy, że:
$$A(X)\cdot B(X) = C(X) = C_4X^4 + C_3X^3 + C_2X^2 + C_1X + C_0$$
oraz mamy oblicznone już każde z $C(0),C(1),C(-1),...$, które możemy rozpisać jako:
$$C(0) = C_0$$
$$C(1) = C_4 + C_3 + C_2 + C_1 + C_0$$
$$C(-1) = C_4 - C_3 + C_2 - C_1 + C_0$$
$$C(2) = 16C_4 + 8C_3 + 4C_2 + 2C_1 + C_0$$
$$C(-2) = 16C_4 - 8C_3 + 4C_2 - 2C_1 + C_0$$
A więc możemy kolejno wyliczać współczynniki Gaussem (w dobrej kolejności lub rozpisać wzroki na pałę jak kto woli) i otrzymać w czasie $\bigO(n)$ wszystkie współczynniki wielomianu $C(X)$. Wiemy teraz, że:
$$C = C_4K^4 + C_3K^3 + C_2K^2 + C_1K + C_0$$
Więc również liczmy to w czasie $\bigO(n)$. Podsumowując złożoność mamy: $T(n) = 5T(\frac{n}{3}) + \bigO(n)$, co z Uniwersalnego twierdzenia o rekurencji daje nam złożoność $\bigO(n^{\log_3 5}) \approx \bigO(n^{1.46})$


