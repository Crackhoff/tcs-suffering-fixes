\section{Grupa B}
Czyli tak zwana grupa pytań średnich.
\subsection{Opisać algorytm Karatsuby mnożenia dużych liczb binarnych}
Algorytm mnożenia Karatsuby opiera się o technikę, dziel i zwyciężaj oraz zauważeniem jednego ciekawego faktu jak możemy oszczędzić jedno mnożenie.
\newline \newline
Idea:
\newline
Na wejściu otrzymujemy dwie liczby A oraz B, n-cyfowe ( mogą być różnych rozmiarów ale to jest tylko kwestia techniczna jak to rozwiązać ), możemy założyć, że są to liczby w systemie binarnym i $n$ jest potęgą dwójki ( jak nie to dopychamy zerami od przodu ).
\newline \newline
Następnie podzielmy nasze liczby na pół (względem długości zapisu) i przedstawmy je jako:

$$A = A_1 \cdot K + A_0 $$
$$B = B_1 \cdot K + B_0 $$
Gdzie $K = 2^{n/2}$.
\newline \newline
Zauważmy teraz, że $$A \cdot B = A_1  B_1  K^2 + A_1  B_0  K + A_0  B_1  K + A_0  B_0$$
Jest to zwykłe mnożenie po rozbiciu $A$ oraz $B$. Zauważmy, że mnożenie przez $K$ jest w czasie $\bigO(n)$, gdyż jest to zwykłe przesunięcie bitowe oraz dodawanie jest również w czasie $\bigO(n)$ więc jedynymi trudnościami są tutaj iloczyny w postaci $A_i B_j$.
\newline \newline
Następną rzeczą, na którą musimy wpać jest jak zaoszczędzić jedno mnożenie.

$$(A_0 + A_1)(B_0 + B_1) = A_0 B_0 + A_0 B_1 + A_1 B_0 + A_1 B_1$$
Z czego wynika

$$(A_0 + A_1)(B_0 + B_1) - A_0 B_0 - A_1 B_1 = A_0 B_1 + A_1 B_0$$

A więc możemy obliczyć rekurencyjnie $A_0 B_0$ oraz $A_1 B_1$. Następnie policzyć rekurencyjnie $(A_0 + A_1)(B_0 + B_1)$ i odjąć od niego $A_0 B_0$ oraz $A_1 B_1$ i otrzymać $A_0 B_1 + A_1 B_0$. Teraz wystarczy podłożyć to do naszego wzoru:
$$A \cdot B = A_1  B_1  K^2 + A_1  B_0  K + A_0  B_1  K + A_0  B_0$$

$$A \cdot B = A_1  B_1  K^2 + (A_1  B_0  + A_0  B_1 ) K + A_0  B_0$$

$$A \cdot B = A_1  B_1  K^2 + ((A_0 + A_1)(B_0 + B_1) - A_0 B_0 - A_1 B_1) K + A_0  B_0$$
I jak widać musimy wykonać tylko 3 różne mnożenia rekurencyjnie.
\newline \newline
Dzięki temu otrzymujemy następującą postać złożoności $T(n) = 3T(\frac{n}{2}) + \bigO(n)$. Co dzięki Uniwersalnemu twierdzeniu o rekurencji mówi nam, że złożoność całego algorytmu wynosi $\bigO(n^{\log_2 3}) \approx \bigO(n^{1.59})$


\subsection{Opisać algorytm Tooma-Cooka mnożenia dużych liczb binarnych}
Na wejściu dostajemy dwie liczby A i B zapisane binarnie, każda o $n$ bitach. Rozbijmy je podobnie jak w algorytmie Karatsuby tylko tym razem z $K = 2^{\frac{n}{3}}$.
$$A = A_2 K^2 + A_1 K + A_0$$
$$B = B_2 K^2 + B_1 K + B_0$$
Potraktujmy teraz A i B jak wielomiany, czyli:
$$A(X) = A_2 X^2 + A_1 X + A_0$$
$$B(X) = B_2 X^2 + B_1 X + B_0$$
Obliczmy teraz wartości wielomianów $A(X)$ i $B(x)$ w punktach $0,1,-1,2,-2$, czyli:
$$A(0) = A_0$$
$$A(1) = A_2 + A_1 + A_0$$
$$A(-1) = A_2 + -A_1 + A_0$$
$$A(2) = 4A_2 + 2A_1 + A_0$$
$$A(-2) = 4A_2 + -2A_1 + A_0$$
Jak widać złożoność obliczania wartości w każdym z tych punktów wynosi $\bigO(n)$, gdyż dodawanie i mnożenie przez 2 i 4 możemy wykonywać w czasie $\bigO(n)$. Wykonajmy również takie same obliczenie dla wielomianu $B(X)$. Skorzystajmy z twierdzenia, że skoro $f,g$ są wielomianami to:
$$f(x) \cdot g(x) = (f\cdot g)(x)$$
A więc jeżeli przez oznaczymy sobie $C(X) = A(X)\cdot B(X)$, ($C(X)$ przy okazji jest wynikiem który chcemy uzyskać) to mamy, że $C(0) = A(0)\cdot B(0), C(1) = A(1)\cdot B(1)$ itd. A więc zauważmy, że każde naszych $A(0),B(0),A(1),...$ ma długość rzedu $\frac{n}{3}$ i obliczmy każde $C(0),C(1),C(-1),...$ rekurencyjnie. A więc jak narazie wykonaliśmy $\bigO(n)$ operacji oraz 5 wywołań rekurencyjnych na liczbach długości $\frac{n}{3}$. Zauważmy, że:
$$A(X)\cdot B(X) = C(X) = C_4X^4 + C_3X^3 + C_2X^2 + C_1X + C_0$$
oraz mamy oblicznone już każde z $C(0),C(1),C(-1),...$, które możemy rozpisać jako:
$$C(0) = C_0$$
$$C(1) = C_4 + C_3 + C_2 + C_1 + C_0$$
$$C(-1) = C_4 - C_3 + C_2 - C_1 + C_0$$
$$C(2) = 16C_4 + 8C_3 + 4C_2 + 2C_1 + C_0$$
$$C(-2) = 16C_4 - 8C_3 + 4C_2 - 2C_1 + C_0$$
A więc możemy kolejno wyliczać współczynniki Gaussem (w dobrej kolejności lub rozpisać wzroki na pałę jak kto woli) i otrzymać w czasie $\bigO(n)$ wszystkie współczynniki wielomianu $C(X)$. Wiemy teraz, że:
$$C = C_4K^4 + C_3K^3 + C_2K^2 + C_1K + C_0$$
Więc również liczmy to w czasie $\bigO(n)$. Podsumowując złożoność mamy: $T(n) = 5T(\frac{n}{3}) + \bigO(n)$, co z Uniwersalnego twierdzenia o rekurencji daje nam złożoność $\bigO(n^{\log_3 5}) \approx \bigO(n^{1.46})$





\subsection{Zapisać i udowodnić chińskie twierdzenie o resztach}
Układ kongruencji
$$x \equiv_{n_1} a_1$$
$$x \equiv_{n_2} a_2$$
$$...$$
$$x \equiv_{n_k} a_k$$
(gdzie $n_i$ i $n_j$ są względnie pierwsze dla $i\neq j$) ma dokładnie jedno rozwiązanie $x<N$ ($N=n_1\cdot n_2 \cdot ... \cdot n_k$)

\subsubsection{Unikalność}
Załóżmy, że dany układ rekurencji ma dwa rozwiązania $x$ i $y$.\\
Zauważmy, że $x\equiv_{n_i} a_i$ oraz $y\equiv_{n_i} a_i$. Zatem otrzymujemy $x-y \equiv_{n_i} 0$.\\
Jako że $n_i$ są parami względne, to zachodzi również $x-y\equiv_N 0$. Ponieważ $x<N$ i $y<N$ to $x-y$ jest wielokrotnością $N$ tylko dla $x=y$.

\subsubsection{Istnienie}
Korzystamy z twierdzenia Bezuta (dla względnie pierwszych liczb całkowitych $x$ i $y$ istnieją liczby całkowite $a$ i $b$ takie że $ax+by=1$).\\
Niech $N_i=N/n_i$ oraz $N_iM_i+n_im_i=1$ (z Bezuta). Wtedy otrzymujemy rozwiązanie kongruencji postaci $x= \sum_{i=1}^{k} a_iN_iM_i$.\\
Czemu to działa? Jak łatwo zauważyć $a_iN_iM_i\equiv_{n_j}0$ dla $j\neq i$, ponieważ $N_i \equiv_{n_j} 0$. Natomiast $a_iN_iM_i\equiv_{n_i} a_i$ gdyż mamy $N_iM_i+n_im_i=1 \Longrightarrow N_iM_i\equiv_{n_i} 1$.



\subsection{Zdefiniować pojęcie ideału, pierścienia ilorazowego oraz pokazać, że $Z_p[X]/(W)$ jest ciałem wtw gdy W jest nierozkładalny}
\subsubsection{Definicja Ideału:}
Jeżeli $R$ jest pierścieniem to ideał $I \subseteq R$ jest ideałem wtw gdy:
\begin{itemize}
    \item $x,y \in I \implies x+y \in I$ (zamknięcie na sumę)
    \item $x \in I, y \in R \implies y\cdot x \in I$ (własność wciągania (jest to silniejsze niż zamkniętość na mnożenie))
\end{itemize}
\subsubsection{Definicja Pierścień ilorazowy:}
Zdefiniujmy sobie relację $x \sim y \iff x - y \in I$, która jest relacja równoważności. Zbiór jej klas abstrakcji jest to $R/I = \{ x + I: x\in R\}$ jest właśnie pierścieniem ilorazowym. Mniej forlmalnie poprostu mówimy, że liczymy modulo ideał $I$.
\subsubsection{Twierdzenie:}
Pierścień $\mathbb{Z}_p[X]/(W)$ jest ciałem wtw. gdy $W$ jest nierozkładalny ( nie ma nietrywialnych dzielników czyli $W$ nie da się przedstawić za pomocą $g_1\cdot g_2$, gdzie $g_1,g_2$ nie są wielomianami stałymi ).
\subsubsection{Dowód:}
($\implies$)\newline
$\mathbb{Z}_p[X]/W$ jest ciałem to załóżmy nie wprost, że $W$ jest rozkładalny. Wynika z tego, że $W$ można przedstawić jako $W = a\cdot b$, gdzie $a,b$ nie są wielomianami stałymi i oba mają mniejszy stopień niż $W$ więc same nie są zerami. A wynika z tego również, że $a\cdot b = 0$ w ciele $\mathbb{Z}_p[X]/W$, czyli są dzielnikami zera. Z czego wynika że nie mogą mieć elementu odwrotnego co prowadzi do sprzeczności.
\newline \newline
($\impliedby$) \newline
Wiemy, że $\mathbb{Z}_p[X]/W$ jest pierścieniem, wystarczy pokazać, że każde $a$ stopnia mniejszego niż $W$ ma odwrotność. Wiemy również z faktu, że $W$ jest nierozkładalny, że $\gcd(a,W) = 1$. Wynika z tego, że możemy zastosować rozszerzony algorytm eulidesa do znajdowania odwrotnośći, gdyż znajdziemy $s,t$, takie, że $a\cdot s + W\cdot t = 1 \implies a\cdot s = 1 \mod W$. Czyli nasze $s$ jest elementem odwrotnym z czego wynika że mamy ciało.


\subsection{Pokazać, że każde ciało skończone musi mieć $p^k$ elementów dla pewnej liczby pierwszej p oraz całkowitego k}
\subsubsection{Twierdzenie 1. (O charakterystyce):}
Charakterystyka ciała skończonego zawsze jest dodatnia i jest liczbą pierwszą.
\subsubsection{Dowód Tw. 1:}
 1. Nie zerowość. \newline \newline
 Nie wprost zakładamy, że nie otrzymujemy 0 (poprzez dodawanie jedynki) a wykonaliśmy więcej niz ilość elementów w ciele dodań 1. Co oznacza, że jakiś element musiał się powtórzyć. Jeżeli jakiś element się powtórzył to z definicji suma między powtórzeniami wynosi 0, gdyż jest ono elementem neutralnym dodawania. Czyli 0 musiało wystąpić sprzeczność.
\newline \newline
 2. Charakterystaka jest liczbą pierwszą.\\ \\
 Załóżmy nie wprost, że charakterystyka jest liczbą złożoną równą $n$. Możemy z tego faktu rozbić $n = pq$ gdzie $p,q > 1$. W takim razie $(1+1+...+1) (n$ razy$) = (1+1+...+1) (q$ razy$) \cdot (1+1+...+1) (p$ razy$)$ z czego wynika, że $p$ lub $q$ jest 0. Co oznacza że $n$ nie jest najmniejsza taką liczbą która po dodaniu $n$ razy 1 otrzymamy 0. Czyli mamy sprzeczność z definicji charakterystyki.
 \subsubsection{Twierdzenie 2.}
 Niech $\mathbb{F}$ będzie ciałem skończonym charakterystyki $p$. Wtedy istnieje takie $k$, że $|\mathbb{F}| = p^k$.
 \subsubsection{Dowód Tw. 2}
Wiemy, że $\mathbb{Z}_p$ jest ciałem. Możemy więc wprowadzić mnożenie przez element z $\mathbb{Z}_p$ elementów z ciała $\mathbb{F}$ zdefiniowane jako $a\cdot x = x + x + ... + x (a razy)$. Wynika z tego, że mamy teraz przestrzeń linową na $Z_p$ gdyż mamy ciało oraz mnożenie przez skalar. Skoro $\mathbb{F}$ jest przestrzenią liniową nad $Z_p$ to ma ona skończony wymiar, nazwijmy go $k$, oraz jakąś baze $x_1,...,x_k$. A więc każdy element $x \in \mathbb{F}$ możemy zapisać jako $a_1x_1 + a_2x_2 + ... + a_kx_k$. Różne ciągi $(a_1,...,a_k)$ dają różne elementy z $\mathbb{F}$ z czego wynika, że liczba elementów $\mathbb{F}$ jest taka sama jak liczba ciągów $(a_1,...,a_k)$ czyli $p^k$.


\subsection{Opisać algorytm faktoryzacji Fermata}
Chcemy rozłożyć liczbę $n$ na czynniki pierwsze. \newline
Założenia: \newline
$n$ - nieparzyste ( inaczej dzielimy przez 2 dopóki możemy )
\newline
$n$ - ma rozkład jakiś rozkład, założmy, że $n = pq$ ($p,q$ nie muszą być pierwsze )
\newline \newline
Załóżmy, bez stray ogólności że $p > q$.
Weźmy sobie teraz $a = \frac{p+q}{2}$ oraz $b = \frac{p-q}{2}$ i zauważmy fakt, że:
$$a^2 - b^2 = (a+b)(a-b) = (\frac{p+q+p-q}{2}) \cdot (\frac{p+q-p+q}{2}) = pq = n$$
A więc jeżeli znajdziemy takie $a$, że $a^2 - n = b^2$, gdzie $b^2$ jest dowolnym kwadratem liczby naturalnej to wyciągniemy z nich informację o $p$ oraz $q$ ( $p = a+b, q = a-b$ ).
\newline \newline
Idea algorytmu:
\begin{itemize}
    \item 1. Weź początkowo $a = \lceil \sqrt{n} \rceil$
    
    \item 2. Sprawdź czy $a^2 - n$ jest kwadratem liczby naturalnej ( binsearch czy cokolwiek ), Jeżeli tak to przewij bo masz a oraz b
    
    \item 3. W przeciwnym wypadku zwiększ $a$ o 1 i wróć do kroku 1.
\end{itemize}
Ciekawa własność tego algorytmu to fakt, że znaleźliśmy dzielnik po $a - \sqrt{n}$ krokach ( gdyż zaczynaliśmy na $\sqrt{n}$ a skonczyliśmy na $a$ ) a więc wykonaliśmy następującą ilość operacji:
$$a - \sqrt{n} = \frac{a^2 - n}{a + \sqrt{n}} = \frac{b^2}{a + \sqrt{n}} \leq \frac{b^2}{\sqrt{n}} \leq  \frac{(p-q)^2}{4\sqrt{n}}$$

Czyli dla $p,q$ blisko siebie działa bardzo szybko a dla $p-q \leq \sqrt[4]{n}$ działa w czasie stałym. Niestety jego pesymistyczna złożoność wynosi $\bigO(n)$, gdyż pesymistycznie z $a$ musimy dojść aż do samego $n$


\subsection{Opisać algorytm DSA}
\subsubsection{Przygotowanie do Algorytmu:}
\begin{itemize}
    \item Wbieramy dwie duże liczby pierwsze $p,q$, takie, że $q \mid p-1$.(Standardowo $q$ ma 256 bitów, a $p$ ma 2048.)
    \item Znajdujemy element $a$, taki, że rząd $a$ modulo $p$ wynosi $q$. (Robimy to losując $g$ oraz podstawiając $a = g^{\frac{p-1}{q}} \mod p$, teraz jeżeli $a \neq 1$, to $a^q = 1 \mod p$, gdyż $q$ jest liczbą pierwszą (A mamy tw. Lagrange)
    \item Podajemy do wiadomości publicznej $p,q,a$ (Są one stałą częścią algorytmu)
\end{itemize}
\subsubsection{Generacja kluczy:}
\begin{itemize}
    \item Wylosuj $x \in \{0,...,p-1\}$, oraz oblicz $y = a^x$.
    \item Wartość $x$ to klucz prywatny.
    \item Wartość $y = a^x$ to klucz publiczny.
\end{itemize}
Możemy zaobserwować, że aby odtworzyć klucz prywatny z klucza publicznego to musimy rozwiązać problem logarytmu dyskretnego.
\subsubsection{Podpisywanie wiadomości:}
\begin{itemize}
    \item Generujemy hash $H$ wiadomości, którą chcemy podpisać (Pamiętamy, że wartości $a,p,q$ są znane oraz mamy nasze klucze).
    \item Losujemy $k$, takie, że $1 < k < q$.
    \item Obliczamy $r = (a^k \mod p) \mod q$.
    \item Obliczamy $s = \frac{H + x\cdot r}{k} \mod q$
    \item Zwróć parę $(r,s)$.
\end{itemize}
\subsubsection{Weryfikacja podpisu:}
\begin{itemize}
    \item Oblicz $\alpha = \frac{H}{s} \mod q$.
    \item Oblicz $\beta = \frac{r}{s} \mod q$.
    \item Oblicz $\gamma = (a^{\alpha}\cdot y^{\beta} \mod p) \mod q$.
    \item Sprawdź czy $\gamma = r$.
\end{itemize}
\subsubsection{Dowód działania:}
Oznaczmy sobie $w = (H + x\cdot r)^{-1} \mod q$ (czyli odwrotność $s$ ale bez $k$ jeszce). Teraz $\alpha = w\cdot k\cdot H \mod q, \beta = w\cdot r \cdot k$. Wiemy, że $a$ ma rząd $q$ modulo $p$ co oznacza, że dla dowolnego $t$ zachodzi $a^t \mod p = a^{t \mod q} \mod p$. Rozpiszmy teraz $\gamma = (a^{\alpha}\cdot y^{\beta} \mod p) \mod q = (a^{w\cdot k\cdot H \mod q} \cdot (a^{x})^{w\cdot r \cdot k} \mod p) \mod q  = a^{k\cdot w \cdot (H + x\cdot r) \mod q}\mod p \mod q$. A z definicji $w$ jest odwrotnością $(H + x\cdot r)$ modulo $q$, czyli Zachodzi $\gamma = (a^k \mod p) \mod q$ co z definicji wynosi $r$. Co należało pokazać.

\subsection{Opisać metodę Baby-Step-Giant-Step}
Owy algorytm był na ASD dlatego to zostawiam, jedynie szybka idea. Dzielimy na pierwiastki i sprowadzamy to do formy $i = a\cdot \sqrt{n} + d, d< \sqrt{n}$ (gdzie $i$ bedzie symbolizowało odpowiednią potęge generatora/podstawy), i wyliczamy wartości dla wszystkich możliwych $d$, a potem iterujemy się po $a$ i sprawdzamy czy istnieje odpowiednie $d$.

\subsection{Opisać kryptosystem plecakowy i uzasadnić, dlaczego nie jest stosowany w praktyce}
Kryptowaluty... Nie to nie o tym ta część :( \newline
\subsubsection{SubSet-Sum Definicja:}
Mamy sobie zbiór liczb $V = \{v_1,v_2,...,v_2\}$ oraz liczbę $s$, stwierdź czy zbiór $A \subseteq V$ taki, że $\sum_{v \in A} v = s$.
\newline
Jak już wiadomo z ASD jest to problem NP-zupełny czyli nie spodziewamy się rozwiązania wielomianowego (no chyba, że P = NP).

\subsubsection{Definicja ciągu nadrosnącego:}
Ciągiem nadrosnącym nazywamy taki ciąg $v_1,v_2,...,v_n$, taki, że dla każdego $i$ zachodzi $v_i > v_1 + v_2 + ... + v_{i-1}$.
\newline
\newline
Jak możemy zauważyć problem SUBSET-SUM jest prostu dla ciągu nadrosnącego ( poprostu robimy zachłana od największych i można prosto udowodnić, że jeżeli możemy wziąc jakiś największy to musimy go wziąć )

\subsubsection{Idea kryptosystemu plecakowego}
\begin{itemize}
    \item Bierzemy sobie nadrosnący ciąg $v_1,...,v_n$, $m > \sum_i v_i$ oraz $a$ względnie pierwsze z $m$.
    \item Konstruujemy ciąg $w_1,...,w_n$, tak, że $w_i = a\cdot v_i \mod m$
    \item Kluczem publicznym jest ciąg $w_1,...,w_n$.
    \item Szyfrowanie: chcąc zaszyforwać $n$-bitową wiadomość $b_1,...,b_n$ (gdzie $b_i$ to $i$-ty bit) wyonujemy $s = \sum_i w_i\cdot b_i$, i wysyłamy $s$.
    \item Deszyfrowanie: mamy $s = \sum_i b_i\cdot w_i$, zauważmy, że z konstrukcji $w_i = v_i \cdot a \mod m$, więc weźmy sobie odwrotność $a$ modulo $m$ i oznaczmy je jako $c$. Wynika z tego, że $s \cdot c = \sum_i b_i \cdot v_i \mod m$. A wiemy z definicji, że $m > \sum_i v_i$, czyli możemy to jednozancznie wyliczyć naszym algorytmem zachłannym.
\end{itemize}

\subsubsection{Dlaczego nie stosujemy kryptosystemu plecakowego:}
Okazuje się, że owy problem jest tylko szczególnym przypadkiem problemu SUBSET-SUM wiec NIE! musi być on NP-zupełny. Co więcej znany jest algorytm wielomianowy go rozwiązujący więc ten problem jest w P(Adi Shamir (1982)).


\subsection{Pokazać, że pierwiastki wielomianu $X^q - X$ dla $q = p^k$ stanowią ciało, podać (inną) praktyczną metodę generowania ciała skończonego}
Jest delikatny blef w pytaniu do tego zadania bo ten wielomian musi być nad ciałem skończonym.
\subsubsection{Twierdzenie}
Dane jest ciało $\mathbb{F}$ charakterystyki $p$, oraz liczba $q = p^k$ dla pewnego $k$. Jeśli wielomian $f(X) = X^q - X$ ma $q$ pierwiastków, to stanowią
one ciało ($q$-elementowe podciało $\mathbb{F}$).
\subsubsection{Dowód}
Niech $A = \{a \in \mathbb{F}: a^q = a\}$ (czyli poprostu nasze pierwiastki z definicji. Pokażemy teraz zamknięcie na operacje, gdyż przemienność, łączność, etc. mamy z operacji na ciele $\mathbb{F}$.
\begin{itemize}
    \item $0 \in A$, gdyż $0^q = 0$.
    \item $1 \in A$, gdyż $1^q = 1$.
    \item $-1 \in A$, gdyż $(-1)^q = -1$ (dla $p = 2$ zachodzi $1 = -1$, gdyż charakterystyka wynosi $2$, czyli $1 + 1 = 0 = 1 + -1$).
    \item Jeżeli $a,b \in A$, to $a\cdot b \in A$, gdyż $(ab)^q = a^qb^q = ab$ czyli jest w $A$. W szczególności mamy stąd, że jeżeli $a \in A$ to $-a \in A$, gdyż $-a = (-1)a$/
    \item Pokażemy, że jeżeli $a,b \in A$, to $a + b \in A$. Ale najpierw udowidnimy sobie coś pomocniczego, czyli że dla elementów z ciała $\mathbb{F}$ zachodzi: $(a+b)^p = a^p +b^p$ dla dowolnych $a,b \in \mathbb{F}$. Rozbijamy z dwumianu Newtona $(a+b)^p = a^p + $ $n \choose 1$ $a^{p-1}b + ... $ $n \choose n-1$ $ab^{p-1} + b^p$. Zauważamy, że każdy z dwumianów na środku jest podzielny przez $p$ (gdyż w liczniku jak rozpiszemy dwumian mamy czynnik $p$, a w mianowniku nie mamy żadnego). Czyli z faktu, że $p$ jest charakterystyką te wyrazy się zerują czyli mamy tezę. Rozpiszmy więc nasze dodawanie i skorzystajmy z faktu, że $q = p^k$. $(a+b)^q = (a+b)^{p^k} = (a+b)^{p\cdot p^{k-1}} = (a^p + b^p)^{p^{k-1}} = (a^{p^2} + b^{p^2})^{p^{k-2}} = ... = a^{p^k} + b^{p^k} = a^q + b^q = a + b$.

\end{itemize}
Generacja innych grup o liczności $p^k$, to poprostu strzelamy w wielomian stopnia $k$ i sprawdzamy czy jest nierozkładalny i mamy na to bardzo dużą szansę.


\subsection{Opisać algorytm Tonellego-Shanksa}
Zacznijmy od kilku twierdzeń:\newline \newline
Twierdzenie 1.\newline
Grupa cykliczna $G$, taka, że $|G| = n = 2m$, ma dokładnie $m$ kwadratów (czyli połowe swojego rozmiaru). Oraz każdy kwadrat ma dokładnie dwa pierwiastki. (co więcej pokażemy, że parzyste potęgi generatora to kwadraty, a nieparzyste nie)
\newline \newline
Dowód:\newline
Weźmy sobie $g$ generator grupy $G$, zauważmy, że każdy element grupy $G$ należy do zbioru $\{g^0,g^1,g^2,...,g^{2m-1}\}$, czyli potęgi generatora wynosą są modulo $2m$. A rozważmy dwa przypadki:
\begin{itemize}
    \item 1. $a = g^k = g^{2j}$, $k$ jest parzyste (czyli jest w parzystą potęgą generatora):
    \newline
    Wtedy możemy zauważyć, że piewiastkami są $g^j,g^{j+m}$, gdyż $(g^{j})^2 = g^{2j} = a$ oraz $(g^{j+m})^2 = g^{2j+2m} = g^{2j} = a$ oraz nie ma żadnego innego.
    \item 2. $a = g^k$, $k$ jest nieparzyste:
    \newline
    Załóżmy teraz nie wprost, że istnieje $b$, ktore jest pierwiastkiem i jest w postaci $b = g^j$, wynika z tego, że $b^2 = g^{2j} = a = g^{k}$, co oznacza, że $2j = k \mod 2m$, co prowadzi do sprzeczności, gdyż $k$ jest nie parzyste a reszta z dzielenia jak i wspołczynnik przez który bierzemy modulo jest parzysty.
\end{itemize}
Czyli udowodniliśmy sobie pierwsze twierdzenie.
\newline \newline
Twierdzenie 2.\newline
Mamy grupę cykliczna $G$, taka, że $|G| = n = 2m$. Element równanie $x^2 = a$, takie, że $x,a \in G$ ma rozwiązanie (co oznacze, że a jest kwadratem) wtedy i tylko wtedy, gdy $a^m = 1$ (dla nie kwadratów $a^m = -1$ ($1,-1$ są to tylko symbole, gdyż grupa $G$ to nie muszą być liczby).
\newline \newline
Dowód:\newline
Mamy przypadki (pokaże dwa reszta idzie analogicznie):
\begin{itemize}
    \item 1. $a$ jest kwadratem:\newline
    Z poprzedniego twierdzenia wynika, że jeżeli $a$ jest kwadratem to jest w postaci $a = g^{2j}$. Wiec z tego wynika, że $a^m = g^{2jm} = g^0$, gdyż potęgi generatora bierzemy modulo $2m$.
    \item 2. $a$ nie jest kwadratem:\newline
    Z poprzedniego Tw wiemy, że $a$ jest w postaci $a = g^j$, gdzie $j$ jest nie parzyste, a więc $g^{mj} = g^m = -1$, gdyż bierzemy potęgi generatora modulo $2m$, a $j$ jest nie parzyste.
\end{itemize}
Idea algorytmu:
Mamy grupę $G$, $|G| = n = 2m$.
\begin{itemize}
    \item 1. $q = m$, $t = n$
    \item 2. wylosuj $z$, które nie jest kwadratem, czyli gdy zajdzie $z^m \neq 1$ (powtarzaj ten krok dopóki dobrze nie wylosujesz)
    \item 3. Dopóki $2 \mid q$ wykonaj $q := \frac{q}{2}$, $t := \frac{t}{2}$. Jeżeli dla nowego $q,t$ $a^qz^t \neq 1$ to $t := t + m$. I powtórz ten krok.
    \item 4. Zwróć $a^{\frac{q+1}{2}}z^{\frac{t}{2}}$
\end{itemize}
Zauważ, że w każdym kroku algorytmu trzymamy niezmiennik, że $a^qz^t = 1$. A więc to co zwróciliśmy $a^{\frac{q+1}{2}}z^{\frac{t}{2}} = x$, i zobaczmy że jest to poprawny wynik. $x^2 = a^{q+1}z^{t} = a\cdot a^qz^t = a$ czyli działa. \newline
Niezmienniki jakie utrzymujemy:
\begin{itemize}
    \item $a^qz^t = 1$
    \item Jeżeli $2^r \mid q$ to $2^{r+1} \mid t$
\end{itemize}
Początkowo oczywiście jest to spełnione.
Krok w niezmiennikach:
\begin{itemize}
    \item Jeżeli $a^{\frac{q}{2}}z^{\frac{t}{2}} = 1$, to trywialnie niezmienniki są spełnione.
    \item W przeciwnym wypadku $a^{\frac{q}{2}}z^{\frac{t}{2}} = -1$ (gdyż jakby się przyjrzeć w rozpisanie tego to zmniejszamy potęgę generatora dwukrotnie wiec może to być tylko $-1$). A z definicji $z$ i Tw 2 wiemy,że $z^m = -1$, więc po operacji $t := \frac{t}{2} + m$ mamy $a^{\frac{q}{2}}z^{\frac{t}{2}}z^m = -1\cdot -1 = 1$, czyli jest ok oraz wiemy, że $m$ jest wielkrotnością $2q$ więc drugi niezmiennik dalej zachodzi.
\end{itemize}
Zauważmy jeszcze, że z Tw 1 połowa elementów nie jest kwadratami więc w punkcie 2. losujemy z prawdopodobieństwem $\frac{1}{2}$ (czyli w oczekiwaniu po stałej liczbie kroków mamy dobre $z$).
\newline \newline
Zauważmy również, że cały algorytm ma $\bigO(\log n)$ iteracji (a w czasie iteracji mamy mnożenia, dzielenia i podnoszenie do potęgi) więc jest on wielomianowy.
