\subsubsection{Definicja Ideału:}
Jeżeli $R$ jest pierścieniem to ideał $I \subseteq R$ jest ideałem wtw gdy:
\begin{itemize}
    \item $x,y \in I \implies x+y \in I$ (zamknięcie na sumę)
    \item $x \in I, y \in R \implies y\cdot x \in I$ (własność wciągania (jest to silniejsze niż zamkniętość na mnożenie))
\end{itemize}
\subsubsection{Definicja Pierścień ilorazowy:}
Zdefiniujmy sobie relację $x \sim y \iff x - y \in I$, która jest relacja równoważności. Zbiór jej klas abstrakcji jest to $R/I = \{ x + I: x\in R\}$ jest właśnie pierścieniem ilorazowym. Mniej forlmalnie poprostu mówimy, że liczymy modulo ideał $I$.
\subsubsection{Twierdzenie:}
Pierścień $\mathbb{Z}_p[X]/(W)$ jest ciałem wtw. gdy $W$ jest nierozkładalny ( nie ma nietrywialnych dzielników czyli $W$ nie da się przedstawić za pomocą $g_1\cdot g_2$, gdzie $g_1,g_2$ nie są wielomianami stałymi ).
\subsubsection{Dowód:}
($\implies$)\newline
$\mathbb{Z}_p[X]/W$ jest ciałem to załóżmy nie wprost, że $W$ jest rozkładalny. Wynika z tego, że $W$ można przedstawić jako $W = a\cdot b$, gdzie $a,b$ nie są wielomianami stałymi i oba mają mniejszy stopień niż $W$ więc same nie są zerami. A wynika z tego również, że $a\cdot b = 0$ w ciele $\mathbb{Z}_p[X]/W$, czyli są dzielnikami zera. Z czego wynika że nie mogą mieć elementu odwrotnego co prowadzi do sprzeczności.
\newline \newline
($\impliedby$) \newline
Wiemy, że $\mathbb{Z}_p[X]/W$ jest pierścieniem, wystarczy pokazać, że każde $a$ stopnia mniejszego niż $W$ ma odwrotność. Wiemy również z faktu, że $W$ jest nierozkładalny, że $\gcd(a,W) = 1$. Wynika z tego, że możemy zastosować rozszerzony algorytm eulidesa do znajdowania odwrotnośći, gdyż znajdziemy $s,t$, takie, że $a\cdot s + W\cdot t = 1 \implies a\cdot s = 1 \mod W$. Czyli nasze $s$ jest elementem odwrotnym z czego wynika że mamy ciało.