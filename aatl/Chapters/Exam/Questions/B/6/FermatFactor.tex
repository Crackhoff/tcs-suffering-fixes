Chcemy rozłożyć liczbę $n$ na czynniki pierwsze. \newline
Założenia: \newline
$n$ - nieparzyste ( inaczej dzielimy przez 2 dopóki możemy )
\newline
$n$ - ma rozkład jakiś rozkład, założmy, że $n = pq$ ($p,q$ nie muszą być pierwsze )
\newline \newline
Załóżmy, bez stray ogólności że $p > q$.
Weźmy sobie teraz $a = \frac{p+q}{2}$ oraz $b = \frac{p-q}{2}$ i zauważmy fakt, że:
$$a^2 - b^2 = (a+b)(a-b) = (\frac{p+q+p-q}{2}) \cdot (\frac{p+q-p+q}{2}) = pq = n$$
A więc jeżeli znajdziemy takie $a$, że $a^2 - n = b^2$, gdzie $b^2$ jest dowolnym kwadratem liczby naturalnej to wyciągniemy z nich informację o $p$ oraz $q$ ( $p = a+b, q = a-b$ ).
\newline \newline
Idea algorytmu:
\begin{itemize}
    \item 1. Weź początkowo $a = \lceil \sqrt{n} \rceil$
    
    \item 2. Sprawdź czy $a^2 - n$ jest kwadratem liczby naturalnej ( binsearch czy cokolwiek ), Jeżeli tak to przewij bo masz a oraz b
    
    \item 3. W przeciwnym wypadku zwiększ $a$ o 1 i wróć do kroku 1.
\end{itemize}
Ciekawa własność tego algorytmu to fakt, że znaleźliśmy dzielnik po $a - \sqrt{n}$ krokach ( gdyż zaczynaliśmy na $\sqrt{n}$ a skonczyliśmy na $a$ ) a więc wykonaliśmy następującą ilość operacji:
$$a - \sqrt{n} = \frac{a^2 - n}{a + \sqrt{n}} = \frac{b^2}{a + \sqrt{n}} \leq \frac{b^2}{\sqrt{n}} \leq  \frac{(p-q)^2}{4\sqrt{n}}$$

Czyli dla $p,q$ blisko siebie działa bardzo szybko a dla $p-q \leq \sqrt[4]{n}$ działa w czasie stałym. Niestety jego pesymistyczna złożoność wynosi $\bigO(n)$, gdyż pesymistycznie z $a$ musimy dojść aż do samego $n$