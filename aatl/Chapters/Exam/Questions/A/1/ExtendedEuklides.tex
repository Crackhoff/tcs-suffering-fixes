Dla danych $a,b > 0$, algorytm zwraca $d = \gcd(a,b)$ oraz takie, że $s,t \in \mathbb{Z}$, że $d = s\cdot a + t\cdot b$. 
\newline

Algorytm Euklidesa
\begin{itemize}
    \item 1. Jeśli $a < b$, zamień $a$ i $b$.

    \item 2. Jeśli $b = 0$, zwróć $d = a$, oraz parę $(1,0)$.
    \item 3. Podziel z resztą $a$ przez $b$, otrzymując $a = q \cdot b + r$.
    \item 4. Wywołaj $\gcd(b,r)$, otrzymując $d$ oraz parę $(s,t)$, taką, że $s\cdot b + t\cdot r = d$
    \item 5. Zwróć $d$ oraz parę $(t,s - t \cdot q)$
\end{itemize}

Dlaczego ten algorytm działa ? \newline
To, że zwraca on poprawne $d$ to chyba już każdy widział wiele razy (pokazuje się, że skoro $d \mid a$ oraz $d \mid b$ to $d \mid (a - b)$, jeżeli $a \geq b$, a to idzie odrazu z przystawania modulo $d$). Jedyne co musimy pokazać, to fakt, że zwraca on poprawne $s,t$, czyli takie że zachodzi $s\cdot a + t \cdot b = d$. A więc założmy sobie niezmiennik, że w każdym kroku algorytmu nasze $s,t$ dla obecnych $a,b$ jest poprawne.\newline \newline

Baza ($b = 0$):\newline
Zwracamy, że $d = a$, oraz zwracamy $(1,0)$, z czego wynika, że $s = 1, t = 0$ czyli $1\cdot a + 0 \cdot b = a = d$. Czyli jest OK
\newline
\newline

Krok: \newline
Mamy, $a,b$, oraz $a = q\cdot b + r$ oraz zwrócone z rekursji ( na argumentach $(b,r)$) $d,s,t$ takie, że $s\cdot b + t\cdot r = d$. Chcemy teraz zmienić nasze $s,t$ tak aby warunek zachodził dla $a,b$. A więc chcemu powiedzieć że dla nas będzie to $x = t$, $y = s - t \cdot q$ (konstruujemy nasze nowe $s,t$). Sprawdźmy teraz czy dla naszego $x,y$ zachodzi nasz warunek. $$x\cdot a + y \cdot b = t \cdot a + (s - t\cdot q) \cdot b = t \cdot a + s \cdot b - t \cdot q \cdot b$$
Tutaj skorzystamy z faktu, że $a = q\cdot b + r$, z czego wynika, że $q\cdot b = a - r$.

$$t \cdot a + s \cdot b - t \cdot q \cdot b = t\cdot a + s\cdot b - t\cdot(a - r) =  t\cdot a + s \cdot b - t \cdot a + t\cdot r = s \cdot b + t \cdot r = d = x\cdot a + y \cdot b$$

Czyli jak widać nasz niezmiennik jest zachowany czyli algorytm działa. \newline

Złożność:
Zauważmy, że każdym przynajmniej jedna z liczb $a,b$ spadnie nam o 2 (casologia):
\begin{itemize}
    \item 1. $b \leq a/2$ z czego wynika, że $a\mod b < b < a/2$ czyli do rekurencji przekazaliśmy argument conajmniej dwukrotnie mniejszy
    
    \item 2. $a > b > a/2$ wynika z tego, że $a - b < a/2$ ( a odejmowanie w tym przypadku zachowuje się tak samo jak modulo, gdyż $a = 1 \cdot b + r$, z czego wynika, że $a - b = 1 \cdot b + r - b = r$ )
\end{itemize}
Czyli mamy $\bigO(\log a)$ (założenie, że $a \geq b$) kroków rekursji co daje nam złożoność $\bigO(\log a \cdot M(a))$, gdzie $M(a)$ to złożonośc operacji w jednym kroku rekurencji (Według wykładu złożoność całkowita to $\bigO(\log^2 a)$ ale nie za bardzo wiem czemu jak w każdym kroku wykonujemy mnożenie,dzielenie i modulo co nie jest tanie chyba że da się go jakość zoopcić).
