Definicja problemu PRIMES:\newline
Mając na wejściu liczbę $p$ stwierdź czy $p$ jest liczbą pierwszą (Odpowiedź TAK/NIE).
\newline \newline
PRIMES $\in coNP$: \newline
Zgadnij $d \in \{2,...,p-1\}$, jeżeli $d \mid p$ odpowiedz NIE.
\newline \newline
PRIMES $\in NP$: \newline
Tw. $\mathbb{Z}_p^*$ ma rząd $p-1$ wtedy i tylko wtedy, gdy $p$ jest pierwsze.\newline
Zgadujemy $g$(generator $\mathbb{Z}_p^*$), oraz rozkład na czynniki pierwsze $p-1$, czyli $p_1^{\alpha_1}\cdot ... \cdot p_s^{\alpha_s} = p-1$. Następnie sprawdzamy:
\begin{itemize}
    \item Rekurencyjnie dla każdego zgadniętego $p_i$ czy jest pierwsze.
    \item Czy $g^{p-1} = 1 \mod p$.
    \item Czy dla każdego $p_i$ zachodzi, że $g^{\frac{p-1}{p_i}} \neq 1 \mod p$ (Zauważ że nie uwzględniamy potęg liczb pierwszych gdyż są to najwieksze dzielniki, w których brakuje dokładnie czynnika więc jeżeli dla jakiegoś mniejszego dzielnika by to zachodziło to dla jednego z tych też zajdzie, gdyż dzieli on jeden z naszych dzielników).
\end{itemize}
Jeżeli wszystkie te warunki są prawdziwe to odpowaiadamy TAK.
\newline \newline
PRIMES $\in BPP$: \newline
Dowód jest przez pokazanie algorytmu Millera-Rabina.

%\newline \newline

PRIMES $\in P$:\newline
Dowód to pokazanie algorytmu AKS którego tutaj raczej nie trzeba będzie pokazać.

%\newline \newline

Definicja problemu FACTORING:\newline
Na wejściu dana jest liczba $n$ oraz liczba $k$. Stwierdź czy istnieje dzielnik $n$ mniejszy lub równy $k$. Czyli formalnie $\exists_d: 2 \leq d \leq k \land d \mid n$.
\newline \newline

FACTORING $\in NP$: \newline
Zgadnij $d \in \{2,...,k\}$ jeżeli $d | n$ odpowiedz TAK.
\newline \newline

FACTORING $\in coNP$: \newline
Zgadujemy rozkład na czynniki pierwsze liczby $n$, czyli mamy $p_1^{\alpha_1}\cdot ... \cdot p_s^{\alpha_s}$. Następnie sprawdzamy czy $p_1^{\alpha_1}\cdot ... \cdot p_s^{\alpha_s} = n$ oraz czy każdego $p_i$ jest pierwsze (albo poprzez AKS albo odpalamy się na algorytmie w $NP$). Jeżeli dla każego $p_i$ nasze sprawdzenie odpowiedziało tak to znajdujemy najmniejsze $p_i$ w naszym rozkładzie i zwracamy TAK, jeżeli najmniejsze $p_i$ jest mniejsze lub równe $k$ w przeciwnym odpowiedz NIE.
\newline
Więcej o tym problemie nie potrafimy narazie powiedzieć.