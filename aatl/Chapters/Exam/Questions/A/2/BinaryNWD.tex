Pseudokod: \\
Binarny Algorytm Euklidesa:
\begin{itemize}
    \item 1. Jeśli $a < b$, zamień $a$ i $b$,
    
    \item 2. Jeśli $b = 0$, zwróć $a$,
    
    \item 3. Jeśli $2 \mid a$ i $2 \nmid b$, zwróć $gcd(a/2, b)$,
    
    \item 4. Jeśli $2 \nmid a$ i $2 \mid b$, zwróć $\gcd(a, b/2)$,
    
    \item 5. Jeśli $2 \mid a$ i $2 \mid b$, zwróć $2 \cdot \gcd(a/2, b/2)$, 
    
    \item 6. Jeśli $2 \nmid a$ i $2 \nmid b$, zwróć $\gcd(b, a - b)$.
\end{itemize}

Jak możemy zauważyć algorytm ten jest bardzo podobny do normalnego algorytmu euklidesa lecz rozważa on wszystkie możliwe parzystości $a$ oraz $b$ w danym kroku algorytmu.
\newline

\textbf{Dlaczego owy algorytm działa?} \newline 

Tutaj musimy się mocno wycaseować. Zauważmy na początek, że przypadek 1 oraz 2 są dokładnie takie same jak w zwykłym euklidesie.\newline 

Następnie weźmy sobie przypadek 3 i 4 na cel. W tych przypadkach jedna z liczb jest podzielna przez 2 a druga nie jest. Wynika z tego, że w ich nwd na pewno nie jest podzielne przez 2, czyli możemy podzielić tą liczbę, która jest podzielna przez 2 i uruchomić się rekurencyjnie i otrzymamy poprawne $\gcd$.


Następny przypadek jest przypadek 5 i wynika z niego, że $2 \mid a$ oraz $2 \mid b$, co oznacza, że w ich nwd możemy uzględnić czynnik $2$ oraz uruchomić się na $a/2$ oraz $b/2$, gdyż wyciągamy ten czynnik przed funkcję $\gcd$ i w ten sposób otrzymujemy poprawne $\gcd$.



Ostanim przypadkiem jest, $2 \nmid a$ oraz $2 \nmid b$. Ale w nim wykonujemy normalny krok z algorytmu euklidesa czyli $\gcd(a,b) = \gcd(b,b-a)$, który też oczywiście jest poprawny.


Złożność Algorytmu:
\newline
Zauważymy, że w każdym z przypadków od 3 do 5, $a$ lub $b$ spada conajmniej dwukrotnie. Jedynm problemem wydaje się przypadek 6. Ale okazuje się, że nie jest to duży problem, gdyż jeżeli $2 \nmid a$ oraz $2 \nmid b$ to $2 \mid b-a$, czyli w wywołaniu rekurencyjnym zajdzie już przypadek od 3 do 5. A więc w conajwyżej dwóch krokach jedna z liczb spadnie dwukrotnie. A więc otrzymujemy złożoność $\bigO(\log (a + b) \cdot M(a,b))$, gdzie $M(a,b)$ to koszt wykonania operacji podziel przez 2, wymnóż razy 2, sprawdź podzielność przez 2, oraz odejmij liczby $a$ oraz $b$ a każdą z tych operacji jesteśmy w stanie wykonać liniowo względem zapisu liczby czyli w czasie $\bigO(\log n)$, czyli całkowita złożność z podliczonymi operacjami wynosi $\bigO(\log^2 (a+b)) = \bigO(\log^2 (a))$, przy założeniu, że $a \geq b$.