Szybkie przypomnienie jak szyfrujemy ( przynajmniej w tym przypadku ) mamy sobie klucz publiczny i klucz prywatny, klucz publiczny udostępniamy i mówimy jak chcesz do mnie coś wysłać to użyj mojego klucza publicznego (no i oczywiście tej samej metody co ja) a ja sobie to odszyfruje moim kluczem prywatnym i przeczytam twoją wiadomość (funkcję szyfrującą oznaczamy zazwyczaj $E(X)$, a deszyfrująca $D(x)$. \newline \newline

RSA Idea:
\begin{itemize}
    \item Wybiersz sobie liczby pierwsze $p,q$, oraz oblicz $N = p\cdot q$, oraz zauważ, że $\phi(N) = (p-1)(q-1)$ (zamiast $\phi(n)$ można wszędzie użyć lcm($p-1,q-1$))
    \item Wybierz $e$ względnie pierwsze z $\phi(N)$ oraz oblicz takie $d$, że $e\cdot d = 1 \mod \phi(N)$ (można tu użyć rozszerzonego algorytmu euklidesa)
    \item Niech $E(x) = x^e \mod N$ oraz $D(x) = x^d \mod N$, czyli nasz klucz publiczny to $(N,e)$
\end{itemize}

Dlaczego deszyfracja działa? \newline
Rząd grupy multiplikatywnej modulo $N$ wynosi $\phi(N)$, a więc dla każdego $x$ względnie pierwszego z $N$ zachodzi $(x^{e})^d = x^{e\cdot d} = x^1$, gdyż $e\cdot d = 1 \mod \phi(N)$ czyli $e\cdot d$ dzieli rząd naszej grupy multiplikatywnej modulo $N$.
\newline \newline

RSA możemy złamać, gdy rozłożymy $N$ na czynniki. Czyli otrzymamy $p,q$. Obliczymy wtedy $\phi(N)$ oraz znajdziemy $d$ (rozszerzonym euklidesem), dla którego zachodzi $e\cdot d = 1 \mod \phi(N)$. Obecnie nie umiemy dobrze rozkładać na czynniki dlatego nie umiemy też łatwo łamać RSA o ile ktoś dobrze dobrał liczby.
\newline \newline

Idea El-Gamal:
\begin{itemize}
    \item Wybierzmy sobie jakąś grupę $G$ (Na przykład grupę multiplikatywną ciała skończonego $\mathbb{F}_q$) oraz element $g \in G$ (najlepiej generator).
    \item Wylosuj liczbę $x \in G$.
    \item Klucz publiczny to $(g,g^x)$, a prywatny to $(g,x)$
    \item Szyfrowanie wygląda w następujący sposób, wylosuj liczbę $y$ oraz oblicz $g^y$ oraz $g^{xy}$ oraz wyślij wiadomość $P$ w postaci $(g^y,P\cdot g^{xy})$. 
\end{itemize}
Znając $x$ oraz $g^y$ możemy wykonać operację $(g^y)^x = g^{xy}$, następnie znaleść odwrotność $g^{xy}$ i odzyskać $P$, gdyż $P\cdot g^{xy} \cdot g^{-xy} = P$
\newline \newline
Problem w odszyfrowaniu polega na tym, że znająć $g^x,g^y$ nie potrafimy obliczyć $g^{xy}$, czyli sprowadza się do to szyfrowanie Diffiego-Hellmana, co rozwiązuje się logarytmem dyskretnym, którego nie potrafimy obecnie szybko liczyć.

