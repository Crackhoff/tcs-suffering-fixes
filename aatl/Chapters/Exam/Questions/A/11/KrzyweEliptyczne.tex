\subsubsection{Definicja Krzywa eliptyczna:}
To zbiór rozwiązań w pewnym ciele $\mathbb{F}$ (punktów $(x,y)$) równania:
$$y^2 = x^3 + ax + b$$
(Jest to krzywa w postaci Weierstrassa i każdą krzywą można do takiej postaci sprowadzić rówanie ogólne wygląda dziko czyli tak: $y^2 + a_1xy + a_3y = x^3 + a_2x^2 + a_4x + a_6$).
\newline \newline
My chcemy aby nasz krzywa była "gładka" i nie miała "ostrza" czyli formanie wyznacznik krzywej $\delta = 4a^3 + 27b^2$ musi być różny od 0.

\subsubsection{Definicja grupy:}
Jeśli $P = (x, y)$ leży na krzywej eliptycznej, to $(x, -y)$ też i oznaczmy go przez $-P$. Jeśli $P$ i $Q$ są punktami na krzywej eliptycznej, to prosta $PQ$ musi
(prawie zawsze!) przeciąć krzywą w jeszcze jednym punkcie $R$.
Definiujemy sumę punktów $P + Q$ jako $-R$ (uwaga, tam jest minus przed $R$!). Aby obsłużyć przypadek $Q = -P$, dodajemy zatem do krzywej jeszcze sztuczny punkt $O$, leżący "w nieskończoności" i
definiujemy $P + (-P) = O$ (a także $P + O = P$). Aby z kolei obliczyć sumę $P + P$, rysujemy styczną do krzywej w
punkcie $P$, znajdujemy jej punkt przecięcia $Q$ z krzywą, i bierzemy $P + P = -Q$, chyba że nie ma takiego $Q$ to zwracamy $O$. (Ogólnie jak nie ma jakiegoś punktu to mówmy w skrócie $O$, za wyjątkiem $P+O$)

\subsubsection{Dzikie wzory:}
Mając $P = (x_P, y_P)$ oraz $Q = (x_Q, y_Q)$, to możemy obliczć $P + Q = S = (x_S , y_S )$ ze wzorów:
$$x_S = \lambda^2 - x_P - x_Q$$
$$y_S = -y_P - \lambda(x_S - x_P)$$

gdzie $\lambda = \frac{3x_{P}^2 +a }{2y_p}$ jeżeli $P = Q$, w przeciwnym wypadku $\lambda = \frac{y_Q - y_P}{x_Q - x_P}$